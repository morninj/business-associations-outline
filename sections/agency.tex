\section{Agency}

\subsection{Who Is an Agent?}

\subsubsection{Definition of Agency: Restatement (Third) \S\S\ 1.01--1.03}

\begin{enumerate}
    \item (Agency is all common law. The Restatement is usually persuasive.)
    \item ``Agency is the \textbf{fiduciary relationship} that arises when one 
    person (a `principal') \textbf{manifests assent} to another person (an `agent') 
    that the agent shall act on the principal's \textbf{behalf} and subject to 
    the principal's \textbf{control}, and the agent manifests assent or 
    otherwise \textbf{consents} so to act.'' \S\ 1.01.
    \item Other definitions of agency are not controlling. \S\ 1.02.
\end{enumerate}

\subsubsection{Control Establishes Agency: \emph{Gorton v. Doty}}
\label{subsub:gorton}

When she loaned her car, the teacher controlled the coach's actions by 
designating the purpose (he could only use it to drive players to the football 
game) and the driver (only he could drive it). That level of control 
established an agency relationship.

\begin{enumerate}
    \item Doty let Garst use her car to drive high school football players to 
    a game.  Gorton was injured in an accident. Gorton sued 
    Doty.\footnote{Casebook p.  1.}
    \item Was Garst an agent of Doty while he was driving the car?
    \begin{enumerate}
        \item Definition of agency: see Restatement (Third) \S\ 1.01 
        (fiduciary relationship, manifestation of assent, on the principal's 
        behalf and subject to his control, and with the agent's 
        consent).\footnote{See also casebook p. 2.}
        \item \textbf{Manifestation of assent}: yes---Doty had a conversation 
        with the coach and gave him the keys.
        \item \textbf{Acting on behalf}: yes, probably---she probably would 
        have driven the players herself (or asked someone else) if the coach 
        hadn't. Doty benefits, arguably. (The dissent disagrees on this 
        point; so does Cable.)
        \item \textbf{Control}: yes---Doty set the purpose (to go to the game) 
        and a condition precedent (that only he could drive).
        \item \textbf{Consent}: yes---the coach consented by driving.
    \end{enumerate}
    \item Other issues:
    \begin{enumerate}
        \item An agency relationship does not require a business matter, a 
        contract, or compensation. It only requires an agreement. (For 
        instance, a donative promise would establish agency, but it would not 
        be enforceable as a contract.)
        \item The court held that ownership of a car establishes a prima facie 
        case against the owner, creating a presumption that the driver is 
        the owner's agent. Doty did not explicitly loan the car to Garst, so 
        Garst was her agent. (Unclear if this is still good law.)
    \end{enumerate}
    \item Affirmed.
    \item Justice Budge, dissenting:
    \begin{enumerate}
        \item ``Agency means more than passive permission. It involves 
        request, instruction or command.''\footnote{Casebook p. 5.} (In other 
        words, there should not be a presumption that the owner of the car is 
        the principal and that the driver is his agent.)
    \end{enumerate}
    \item How could Doty have avoided liability?\footnote{Casebook p. 6 q. 3.}
    \begin{enumerate}
        \item \textbf{Insurance}.
        \item \textbf{Legal structuring}:
        \begin{enumerate}
            \item Get compensation from the coach (which would mean that he 
            was no longer acting on her behalf; e.g., when you rent a car, 
            you're not driving on Hertz's behalf).
            \item Explicitly say that this arrangement is a ``loan'' in her 
            conversation.
            \item Write a letter of understanding or get a signed agreement.
            \item Avoid taking control (i.e., let the coach do what he wants 
            with the car).
            \item Loan the car to the school principal and let him decide what 
            to do with it (which would negate the ``control'' and ``acting on 
            behalf'' elements).
        \end{enumerate}
    \end{enumerate}
\end{enumerate}

\subsubsection{Creditor as Principal: \emph{A. Gay Jenson Farms Co. v. 
Cargill, Inc.}}
\label{subsub:cargill}

A creditor that exercises enough control over a debtor's business can be 
liable as a principal for the debtor's acts.

\begin{enumerate}
    \item Actors: Cargill (principal), Warren (agent), farmers (third 
    parties).
    \item Farmers sold grain on trade credit\footnote{E.g., give the buyer 
    \$100,000 of grain on credit and let him repay it in \$5,000 increments.} 
    to Warren, a local grain elevator operator and grain reseller. Warren 
    bought the grain on trade credit and sold it to Cargill. Cargill (1) bought 
    grain from Warren and (2) loaned Warren money.
    \item Warren entered into a financing agreement with Cargill.
    Warren defaulted, owing \$2 million to the farmers and \$3.6 million to 
    Cargill.
    \item The farmers sued Warren and Cargill (the one with the deep pockets) 
    to recover money for the grain they sold on trade 
    credit.\footnote{Casebook p. 7--9.}
    \item Did Cargill's dealings with Warren cause it to become liable as a 
    principal on contracts Warren made with the farmers?\footnote{Casebook p. 
    9.}
    \begin{enumerate}
        \item \textbf{Manifestation of assent}: yes---the relationship was 
        carefully documented.
        \item \textbf{On behalf}: yes---Warren sold almost all of its grain to 
        Cargill.\footnote{Casebook p. 11.} Almost all of Warren's actions 
        directly benefited Cargill.
        \item \textbf{Control}: yes---Cargill directly made business decisions 
        for Warren. Cargill's name was even alongside Warren's on Warren's 
        checks.
        \item \textbf{Consent}: yes.
    \end{enumerate}
    \item It's possible for a creditor to be liable as principals for the acts 
    of a debtors---for instance, by taking over management of the debtor's 
    business.\footnote{Casebook p. 10--11.}
    \item Cargill's relationship with Warren was more than that of a buyer and 
    supplier (no ``independent business,'' all financing from Cargill, almost 
    all business with Cargill), and more than that of a creditor and debtor 
    (``aggressive financing,'' ``unique fabric''). Rather, Cargill had a 
    ``paternalistic relationship'' with Warren that elevated their 
    relationship to that of a principal and agent.\footnote{Casebook p.  
    11--12.}
    \item This situation was different from ordinary bank financing because of 
    the level of control Cargill exercised.
    \item How could Cargill have avoided liability?
    \begin{enumerate}
        \item To avoid the risks of default and opportunistic behavior, 
        Cargill's lawyers did:
        \begin{enumerate}
            \item Audit Warren's accounting.
            \item Assert operational control, at least on big purchases, etc.
            \item Make Warren use Cargill's business forms.
            \item Secure collateral, via the right of first refusal on 
            Warren's grain sales.
        \end{enumerate}
        \item But: as this case shows, excessive control can create agency 
        liability. Aggressive lawyering can create as well as mitigate risk.
    \end{enumerate}
\end{enumerate}

\subsection{Liability of Principal to Third Parties in Contract}

\subsubsection{The Agent's Authority}

\paragraph{Actual and Apparent Authority: Restatement (Third) \S\S\ 
2.01--2.03}

\begin{enumerate}
    \item (These types often overlap.)
    \item An agent acts with \textbf{actual authority} when he 
    \textbf{reasonably believes} that the principal wants him so to act. \S\ 
    2.01.
    \item The agent has actual authority to take actions that the principal 
    \textbf{designated or implied}, and to take actions that are 
    \textbf{necessary and incidental} to achieving the principal's objectives 
    (e.g., the principal hires a manager, who then hires a janitor).  The 
    reasonableness of the agent's interpretation of the principal's 
    manifestations and objectives can be inferred from context. \S\ 2.02.
    \item \textbf{Apparent authority} is the power of an agent or other actor 
    to affect the principal's legal relations with a \textbf{third party} when 
    the third party \textbf{reasonably believes} that the actor has authority 
    to act on behalf of the principal, and the belief is traceable to the 
    principal's manifestations. \S\ 2.03.
\end{enumerate}

\paragraph{Implied and Apparent Authority: \emph{Mill Street Church of Christ 
v. Hogan}}
\label{par:mill}
~\\\\
Principals can be liable when the agent acts with implied authority. 
Principals can also be liable when a third party relies on the agent's 
apparent authority.

\begin{enumerate}
    \item Actors: church (principal), Bill Hogan (agent), Sam Hogan (third 
    party).
    \item Mill Street Church hired Bill Hogan to paint the church. It had 
    hired Bill several times to fix up its building. It had allowed him to 
    hire his brother Sam several times.  Although the Church did not expressly 
    authorize it this time, Bill hired Sam to help. After a half hour of work, 
    Sam fell and broke his arm.\footnote{Casebook p. 14--15.}
    \item Sam sued the Church and its insurance company for workers' 
    compensation benefits.
    \item Did Bill have authority to hire Sam?
    \item \textbf{Implied authority}: actual authority that the principal 
    intended to give to the agent, which is necessary for the job and can be 
    inferred from the circumstances.\footnote{Casebook p. 15.} See 
    Restatement (Third) \S\ 2.01.
    \item \textbf{Apparent authority}: a third party reasonably relies on the 
    appearance of the agent's actual authority, which the third party can 
    trace to the principal's manifestation. See Restatement (Third) \S\ 2.01.
    \item Held: there was no actual authority because the Church had told Bill 
    to hire Petty, not Sam. There was implied authority because Bill had often 
    hired Sam in the past and he needed assistance for this job. There was 
    also apparent authority because Sam believed Bill had the authority to 
    hire him (and he relied on that belief), and the Church had manifested 
    that Bill had this authority because this was its practice in the past and 
    because it paid Bill for Sam's work---a \textbf{prior course of 
    dealing}.\footnote{Casebook p. 16.}
    \item So, Sam was within the Church's employment when he was injured, so 
    he can recover workers' comp.
\end{enumerate}

\paragraph{Apparent Authority vs. Implied Authority}

\begin{enumerate}
    \item Can you have apparent authority without implied authority?
    \item Yes---this can happen when the principal explicitly tells the agent 
    not to do something, but the agent still does it. There is no claim for 
    implied authority, but the agent may still have acted with apparent 
    authority, and the principal would still be liable.
    \item This outcome seems somewhat unfair, but the policy is this: there is 
    a harm associated with the agent's actions, and the deep-pocketed 
    principal is in a better position to compensate the injured party.
\end{enumerate}

\paragraph{Apparent Authority of Salesmen: \emph{Three-Seventy Leasing 
Corporation v. Ampex Corporation}}
~\\\\
If a salesman has apparent authority to make a sale, the buyer can enforce the 
sales contract against the company. Here, the principal failed to mitigate or 
disavow the third party's reasonable understanding of the salesman's 
authority.

% TODO Was there implied authority in this case? (AC asked the question but 
% never answered it)

\begin{enumerate}
    \item Actors: Ampex (principal), Kays (agent), Joyce (third party).
    \item Joyce (a middleman, like Warren in \emph{Cargill)} ran 370, a 
    company that bought computer hardware from manufacturers and leased it to 
    end users. Kay was a salesman at Ampex, a hardware manufacturer, and a 
    friend of Joyce's. Joyce was negotiating (1) a lease agreement with EDS 
    (an end-user) and (2) a purchase of computer memory from Ampex, which he 
    planned to lease to EDS.\footnote{Casebook p.  16--17.}
    \item Ampex sent Joyce a sales document with spaces for signatures from 
    Joyce and an Ampex officer. Joyce signed and returned the agreement. Ampex 
    apparently never signed it.\footnote{Casebook p. 17.} Ampex apparently 
    breached the contract, the Joyce sued for its enforcement.
    \item Joyce argued that the document was an offer to sell, which he 
    accepted by signing. Ampex argued that it was a solicitation that became 
    an offer to buy when Joyce executed it, which Ampex rejected.
    \item The court held that it was \emph{not} an offer to sell because Ampex 
    lacked the requisite intent.\footnote{Casebook p. 17.} Joyce's signed 
    document ``at most constituted an offer by him to 
    purchase.''\footnote{Casebook p. 18.}
    \item So, did Ampex accept Joyce's offer to buy?
    \item Yes---there was an acceptance. Was Kays authorized to accept? 
    Yes---Kays had apparent authority to accept Joyce's offer on Ampex's 
    behalf because he was a salesman and was designated as the single point of 
    contact for the relationship.\footnote{For the standard for apparent 
    authority, see Restatement (Third) \S\ 2.03, above.} Ampex could have 
    easily crafted the agreement so that another officer's approval was 
    required before the sale was complete. Under these circumstances, ``Joyce 
    could reasonably expect that Kays would speak for the 
    company.''\footnote{Casebook p. 19.}
    \item Was Kays's apparent authority traceable to the principal? It's 
    somewhat mysterious here. This case boils down to the general nature of a 
    salesperson's authority.
\end{enumerate}

\paragraph{Liability of Undisclosed Principal: Restatement (Third) \S\ 2.06}

\begin{enumerate}
    \item \textbf{Generally}: the principal is not bound if A acts outside his 
    authority---but there are important exceptions: (1) the undisclosed 
    principal, (2) ratification, (3) estoppel.
    \item (1) An undisclosed principal is liable to a third party (when the 
    agent acts without actual authority) if the principal \textbf{had notice} 
    of the agent's conduct and \textbf{did not take reasonable steps to 
    notify} the third party.
    \item (2) The undisclosed principal cannot escape liability by relying on 
    instructions he gave to the agent that would have reduced the agent's 
    authority to less than what a third party would reasonably believe the 
    agent to have.
\end{enumerate}

\paragraph{Contracts with Third Parties: Restatement (Third) \S\ 6.01-03}

\begin{enumerate}
    \item When an agent with actual or apparent authority for a 
    \textbf{disclosed principal} makes a contract, the principal and third 
    party are parties to the contract, and the agent is not (unless otherwise 
    agreed).
    \item When an agent with actual or apparent authority for an 
    \textbf{unidentified principal} makes a contract, the principal and third 
    party are parties to the contract, and the agent is not (unless otherwise 
    agreed). (I.e., it's the same as with a disclosed principal.)
    \item When an agent with actual or apparent authority for an
    \textbf{undisclosed principal} makes a contract, (1) the principal is a 
    party, (2) the agent and third party are parties, (3) the principal and 
    third parties have the same relationship as if they had entered the 
    contract directly.
\end{enumerate}

\paragraph{Undisclosed Principals: \emph{Watteau v. Fenwick}}
\label{par:watteau}
~\\\\
An undisclosed principal is liable for acts that are ``within the authority 
usually confided'' to agents.  The Restatement (Second) codified this case's 
rule in the vaguely defined concept of ``inherent agency 
power.''\footnote{Casebook p. 22.} But the Restatement (Third) rejected it in 
favor of the rule that the undisclosed principal is liable only if he had 
\textbf{notice} that the agent had exceeded his authority. See \S\ 2.06. So, 
the defendants in \emph{Watteau} would not be liable under the Restatement 
(Third).\footnote{But Cable disagrees, arguing that \S\ 2.06(2) \emph{does} 
codify \emph{Watteau}.}

\begin{enumerate}
    \item Actors: business owners (principals), Humble (agent), cigar sellers 
    (third parties).
    \item Humble managed a beerhouse for its owners. The license was in his 
    name and his name was painted over the door (so the principal, Humble, was 
    undisclosed). He bought cigars on credit, even though the owners had 
    forbidden him from doing so. The cigar sellers were not aware that Humble 
    was not the owner. So, the owners were potentially \textbf{undisclosed 
    principals}.\footnote{Casebook p. 20--21.}
    \item Rule: an undisclosed principal is liable for acts that are ``within 
    the authority usually confided'' to agents.\footnote{Casebook p. 21.} 
    ``Holding out'' of the agent's authority by the principal is not 
    necessary.\footnote{Casebook p. 22.} Otherwise, undisclosed principals 
    could never be liable.
    \item Held: defendants/owners were liable.
\end{enumerate}

\subsubsection{Ratification}

\paragraph{\emph{Botticello v. Stefanovicz}}
\label{par:botticello}
~\\\\
Ratification is ``the affirmance by a person of a prior act which did not bind 
him but which was done or professedly done on his account.''\footnote{Casebook 
p. 26.} Receiving the benefits of an agreement does not count as ratification.

\begin{enumerate}
    \item Actors: Mary (principal), Walter (agent), Botticello (third party).
    \item Mary and Walter, husband and wife, owned property as tenants in 
    common. Botticello wanted to buy the property. Mary said she wouldn't sell 
    for less than \$85,000. Without Mary's knowledge, Walter conveyed the 
    whole property to Botticello in a lease agreement with an option to buy 
    for \$85,000. He moved in and made improvements, but Mary thought he was 
    just a renter. When Botticello tried to exercise his option to purchase, 
    Mary and Walter refused to honor it.\footnote{Casebook p. 24--25.}
    \item Was there an agency relationship between Mary and Walter?
    \begin{enumerate}
        \item \textbf{Manifestation of assent}: no---Mary had not given 
        express or implied assent to delegating power to Walter.
        \item \textbf{On behalf}: yes.
        \item \textbf{Control}: maybe---she had set a minimum price, but did 
        not specify other terms.
        \item \textbf{Consent}: yes.
    \end{enumerate}
    \item Held: there was no agency relationship.\footnote{Casebook p. 26.} 
    Generally, there is no agency relationship in marriage or in tenancies in 
    common.
    \item Did Mary ratify the agreement by her conduct?
    \begin{enumerate}
        \item Ratification requires (1) \textbf{intent to ratify} and (2) 
        \textbf{full knowledge} of material circumstances.\footnote{Casebook 
        p. 26.}
        \item No: Mary did not intend to ratify the agreement, nor did she 
        have full knowledge of the circumstances. Receiving the benefits of an 
        agreement does not count as ratification.
    \end{enumerate}
    \item Held: Botticello cannot recover against Mary, but he can recover 
    against Walter for specific performance or damages.
    \item (Botticello should have (1) checked the property interest that 
    Walter was selling, and (2) checked the public land records.)
\end{enumerate}

\subsubsection{Estoppel}

\paragraph{Estoppel to Deny Existence of Agency Relationship: Restatement 
(Third) \S\ 2.05}

\begin{enumerate}
    \item If not otherwise liable, a potential principal can still be liable 
    to a third party if he (1) intentionally or carelessly caused the third 
    party to believe he was the principal, or (2) had notice of such a belief 
    and that it might cause a third party to change his position, and did not 
    take reasonable steps to notify him.
\end{enumerate}

\paragraph{\emph{Hoddeson v. Koos Bros.}}
~\\\\
A person has a duty to take reasonable steps to prevent third parties from taking 
detrimental steps on the belief that an agent is acting on that person's 
behalf.

\begin{enumerate}
    \item Hoddeson visited the Koos. Bros. store to buy furniture. She paid 
    \$168.50 to a salesman who promised the furniture would be delivered. It 
    was never delivered, and the store told Hoddeson that the salesman had 
    been an impostor.\footnote{Casebook p. 28--29.}
    \item Apparent authority requires a manifestation from the principal. 
    There was no manifestation here, so there was no apparent authority.
    \item However, the store had a duty to take reasonable steps to prevent 
    deceptive salesmen from defrauding its customers. It breached that duty, 
    so it was liable for Hoddeson's loss.
\end{enumerate}

\subsection{Liability of Principal to Third Parties in Tort}

\subsubsection{Servant Versus Independent Contractor}

\paragraph{Overview}

\begin{enumerate}
    \item Is an operator an employee/servant or independent 
    contractor?\footnote{Casebook p. 35.}
    \item If he was an independent contractor, was he an agent or 
    non-agent?\footnote{Casebook p. 35--36.}
\end{enumerate}

\paragraph{Gas Station I: \emph{Humble Oil \& Refining Co. v. Martin}}
~\\\\
A high degree of control can establish a master-servant relationship. Cf. 
\emph{Hoover} below.

\begin{enumerate}
    \item Schneider operated a gas station by agreement with Humble. Manis was 
    Schneider's employee. Manis's watch, a car parked at the station rolled 
    away and injured Martin and his daughters.\footnote{Casebook p. 36.}
    \item Humble argued that it was not liable because Schneider was an 
    independent contractor.
    \item The agreement between Humble and Schneider gave Humble a high degree 
    of control. It ``required Schneider in effect to do anything Humble might 
    tell him to do.'' Schneider was Humble's servant.\footnote{Casebook p. 37.}
\end{enumerate}

\paragraph{Gas Station II: \emph{Hoover v. Sun Oil Company}}
~\\\\
The operator is more likely to be an independent contractor (and thus not 
liable) if (1) he controls the daily operations and (2) he assumes the risks 
of profit and loss.

\begin{enumerate}
    \item Barone operated a Sun gas station. One of Barone's employees started 
    a fire that injured Hoover.\footnote{Casebook p. 38.}
    \item Sun gave extensive advice to Barone, but ``Barone was under no 
    obligation to follow the advice.'' There were no reporting requirements, 
    and Barone alone assumed the risk of profit or loss.\footnote{Casebook p. 
    39.}
    \item Sun and Barone had a mutual interest, but Sun did not control 
    Barone. ``Sun had no control over the details of Barone's day-to-day 
    operation. Therefore, no liability can be imputed to Sun from the 
    allegedly negligent acts of'' Barone's employee.\footnote{Casebook p. 40.}
\end{enumerate}

\paragraph{Typical Franchise Agreement: \emph{Murphy v. Holiday Inns, Inc.}}
~\\\\
A ``typical franchise agreement'' does not establish a master-servant 
relationship because it does not give the franchisor control over day-to-day 
operations. Cf. \emph{McDonald's} below.

\begin{enumerate}
    \item Betsy-Len ran a Holiday Inn. Murphy slipped and fell in Betsy-Len's 
    building.\footnote{Casebook p. 41--42.}
    \item Murphy argued that Holiday Inn's authority and control over 
    Betsy-Len established an agency relationship. Holiday Inn argued that it 
    had no relationship other than a licensing agreement that allowed 
    Betsy-Len to use the Holiday Inn ``system.''
    \footnote{Casebook p. 42--43.}
    \item Held: this was a ``typical franchise contract'' in which the 
    agreement allowed for ``standardization of business identity, uniformity 
    of commercial service, and optimum public good will,'' but it ``did not 
    give defendant control over the day-to-day operation of Betsy-Len's 
    motel.''\footnote{Casebook p. 44.}
\end{enumerate}

\subsubsection{Tort Liability and Apparent Agency}

\paragraph{\emph{Miller v. McDonald's Corp.}}

\begin{enumerate}
    \item 3K operated a McDonald's franchise. Miller sued after biting into a 
    sapphire in a Big Mac.\footnote{Casebook p. 47.} This case was on appeal 
    from a motion for summary judgment.
    \item The franchise agreement established very tight controls over the 
    daily operations of the franchise, establishing a triable issue of fact as 
    to whether an actual agency relationship existed.\footnote{Casebook p. 
    48--49.}
    \item Miller also argued a theory of apparent agency.
    \begin{enumerate}
        \item Apparent agency exists when (1) the principal holds the party 
        out as an agent and (2) the plaintiff relied on that holding 
        out.\footnote{Casebook p. 50--51.}
        \item McDonald's conceded that it held out 3K as its agent, but it 
        argued that there was insufficient evidence to show that Miller 
        justifiably relied on that holding out.\footnote{Casebook p. 50.}
        \item Held: a jury could find that Miller reasonably believed that all 
        McDonald's restaurants were the same.\footnote{Casebook p. 50.}
    \end{enumerate}
\end{enumerate}

\subsubsection{Scope of Employment}

\paragraph{\emph{Ira S. Bushey \& Sons, Inc. v. United States}}

\begin{enumerate}
    \item % TODO 52
\end{enumerate}

% TODO restatement third:
    % 2.05
    % 2.06
    % 6.01-6.03
    % 2.04, 4.01, 4.02, 4.05(1), 6.10, 7.07, 8.01-05, 8.08-09
% rest second: 220
