\section{Agency}

\subsection{Who Is an Agent?}

\subsubsection{Definition of Agency: Restatement (Third) \S\S\ 1.01--1.03}

\begin{enumerate}
    \item (Agency is all common law. The Restatement is usually persuasive.)
    \item ``Agency is the \textbf{fiduciary relationship} that arises when one 
    person (a `principal') \textbf{manifests assent} to another person (an `agent') 
    that the agent shall act on the principal's \textbf{behalf} and subject to 
    the principal's \textbf{control}, and the agent manifests assent or 
    otherwise \textbf{consents} so to act.'' \S\ 1.01.
    \item Other definitions of agency are not controlling. \S\ 1.02.
\end{enumerate}

\subsubsection{Actual and Apparent Authority: Restatement (Third) \S\S\ 
2.01--2.03}

\begin{enumerate}
    \item An agent acts with \textbf{actual authority} when he 
    \textbf{reasonably believes} that the principal wants him so to act. \S\ 
    2.01.
    \item The agent has actual authority to take actions that the principal 
    \textbf{designated or implied}, and to take actions that are 
    \textbf{necessary and incidental} to achieving the principal's objectives. 
    The reasonableness of the agent's interpretation of the principal's 
    manifestations and objectives can be inferred from context. \S\ 2.02.
    \item \textbf{Apparent authority} is the power of an agent or other actor 
    to affect the principal's legal relations with a \textbf{third party} when 
    the third party reasonably believes that the actor has authority to act on 
    behalf of the principal, and the belief is traceable to the principal's 
    manifestations. \S\ 2.03.
\end{enumerate}

\subsubsection{Control Establishes Agency: \emph{Gorton v. Doty}}
\label{subsub:gorton}

When she loaned her car, the teacher controlled the coach's actions by 
dictating the purpose (he could only use it to drive players to the football 
game) and by setting a condition precedent (only he could drive it). That 
level of control established an agency relationship.

\begin{enumerate}
    \item Doty let Garst use her car to drive high school football players to 
    a game.  Gorton was injured in an accident. Gorton sued 
    Doty.\footnote{Casebook p.  1.}
    \item Was Garst an agent of Doty while he was driving the car?
    \begin{enumerate}
        \item Definition of agency: see Restatement (Third) \S\ 1.01 
        (fiduciary relationship, manifestation of assent, on the principal's 
        behalf and subject to his control, and with the agent's 
        consent).\footnote{See also casebook p. 2.}
        \item \textbf{Manifestation of assent}: yes---Doty had a conversation 
        with the coach and gave him the keys.
        \item \textbf{Acting on behalf}: yes, probably---she probably would 
        have driven the players herself (or asked someone else) if the coach 
        hadn't.
        \item \textbf{Control}: yes---Doty set the purpose (to go to the game) 
        and a condition precedent (that only he could drive).
        \item \textbf{Consent}: yes---the coach consented by driving.
    \end{enumerate}
    \item Other issues:
    \begin{enumerate}
        \item An agency relationship does not require a business matter, a 
        contract, or compensation. It only requires an agreement. (For 
        instance, a donative promise would establish agency, but it would not 
        be enforceable as a contract.)
        \item The court held that ownership of a car establishes a prima facie 
        case against the owner, creating a presumption that the driver is 
        the owner's agent. Doty did not explicitly loan the car to Garst, so 
        Garst was her agent. (Unclear if this is still good law.)
    \end{enumerate}
    \item Affirmed.
    \item Justice Budge, dissenting:
    \begin{enumerate}
        \item ``Agency means more than passive permission. It involves 
        request, instruction or command.''\footnote{Casebook p. 5.} (In other 
        words, there should not be a presumption that the owner of the car is 
        the principal and that the driver is his agent.)
    \end{enumerate}
    \item How could Doty have avoided liability?\footnote{Casebook p. 6 q. 3.}
    \begin{enumerate}
        \item \textbf{Insurance}.
        \item \textbf{Legal structuring}:
        \begin{enumerate}
            \item Get compensation from the coach (which would mean that he 
            was no longer acting on her behalf; e.g., when you rent a car, 
            you're not driving on Hertz's behalf).
            \item Explicitly say that this arrangement is a ``loan'' in her 
            conversation.
            \item Write a letter of understanding or get a signed agreement.
            \item Avoid taking control (i.e., let the coach do what he wants 
            with the car).
            \item Loan the car to the school principal and let him decide what 
            to do with it (which would negate the ``control'' and ``acting on 
            behalf'' elements).
        \end{enumerate}
    \end{enumerate}
\end{enumerate}

\subsubsection{Creditor as Principal: \emph{A. Gay Jenson Farms Co. v. 
Cargill, Inc.}}
\label{subsub:cargill}

A creditor that exercises enough control over a debtor's business can be 
liable as a principal for the debtor's acts.

\begin{enumerate}
    % FIXME four agency factors
    \item Farmers sold grain to Warren, a local grain elevator operator and 
    grain reseller. Warren entered into a financing agreement with Cargill. 
    After Warren defaulted, the farmers sued Warren and Cargill to recover 
    money that Warren owed them for grain.\footnote{Casebook p. 7--9.}
    \item Did Cargill's dealings with Warren cause it to become liable as a 
    principal on contracts Warren made with the farmers?\footnote{Casebook p. 
    9.}
    \item It's possible for a creditor to be liable as principals for the acts 
    of a debtors---for instance, by taking over management of the debtor's 
    business.\footnote{Casebook p. 10--11.}
    \item Cargill's relationship with Warren was more than that of a buyer and 
    supplier, and more than that of a financier. Rather, Cargill had a 
    ``paternalistic relationship'' with Warren that elevated their 
    relationship to that of a principal and agent.\footnote{Casebook p. 
    11--12.}
    \item This situation was different from ordinary bank financing because of 
    the level of control Cargill exercised.
\end{enumerate}

\subsection{Liability of Principal to Third Parties in Contract}

\subsubsection{The Agent's Authority}

\paragraph{Implied and Apparent Authority: \emph{Mill Street Church of Christ 
v. Hogan}}
~\\\\
Principals can be liable when the agent acts with implied authority. 
Principals can also be liable when a third party relies on the agent's 
apparent authority.

\begin{enumerate}
    % FIXME four agency factors
    \item Mill Street Church had hired Bill Hogan several times to fix up its 
    building. It had allowed him to hire his brother Sam several times. 
    Although the Church did not expressly authorize it this time, Bill hired 
    Sam to help. After a half hour of work, Sam fell and broke his 
    arm.\footnote{Casebook p. 14--15.}
    \item Sam sued the Church and its insurance company for workers' 
    compensation benefits.
    \item Did Bill have authority to hire Sam?
    \item \textbf{Implied authority}: actual authority that the principal 
    intended to give to the agent, which is necessary for the job and can be 
    inferred from the circumstances.\footnote{Casebook p. 15.} See 
    Restatement (Third) \S\ 2.01.
    \item \textbf{Apparent authority}: a third party relies on the appearance 
    of the agent's actual authority, which the principal has manifested. See 
    Restatement (Third) \S\ 2.01.
    \item Held: there was implied authority because Bill had often hired Sam 
    in the past and he needed assistance for this job. There was also apparent 
    authority because Sam believed Bill had the authority to hire him (and he 
    relied on that belief), and the Church had manifested that Bill had this 
    authority because this was its practice in the past and because it paid 
    Bill for Sam's work.\footnote{Casebook p. 16.}
    \item So, Sam was within the Church's employment when he was injured, so 
    he can recover workers' comp.
\end{enumerate}

\paragraph{Apparent Authority of Salesmen: \emph{Three-Seventy Leasing 
Corporation v. Ampex Corporation}}
~\\\\
If a salesman has apparent authority to make a sale, the buyer can enforce the 
sales contract against the company.

\begin{enumerate}
    % FIXME four agency factors
    \item Joyce ran 370, a company that bought computer hardware from 
    manufacturers and leased it to end users. Kay was a salesman at Ampex, a 
    hardware manufacturer, and a friend of Joyce's. Joyce was negotiating (1) 
    a lease agreement with EDS (an end-user) and (2) a purchase of computer 
    memory from Ampex, which he planned to lease to EDS.\footnote{Casebook p. 
    16--17.}
    \item Ampex sent Joyce a sales document with spaces for signatures from 
    Joyce and an Ampex officer. Joyce signed and returned the agreement. Ampex 
    apparently never signed it.\footnote{Casebook p. 17.} Ampex apparently 
    breached the contract, the Joyce sued for its enforcement.
    \item Joyce argued that the document was an offer to sell, which he 
    accepted by signing. Ampex argued that it was a solicitation that became 
    an offer to buy when Joyce executed it, which Ampex rejected.
    \item The court held that it was \emph{not} an offer to sell because Ampex 
    lacked the requisite intent.\footnote{Casebook p. 17.} Joyce's signed 
    document ``at most constituted an offer by him to 
    purchase.''\footnote{Casebook p. 18.}
    \item So, did Ampex accept Joyce's offer to buy?
    \item Yes---Kays had apparent authority to accept Joyce's offer on Ampex's 
    behalf because he was a salesman and was designated as the single point of 
    contact for the relationship. Ampex could have easily crafted the 
    agreement so that another officer's approval was required before the sale 
    was complete. Under these circumstances, ``Joyce could reasonably expect 
    that Kays would speak for the company.''\footnote{Casebook p. 19.}
\end{enumerate}

\paragraph{\emph{Watteau v. Fenwick}}

\begin{enumerate}
    \item % FIXME 20
\end{enumerate}

\subsubsection{Ratification}

\paragraph{\emph{Botticello v. Stefanovicz}}

\begin{enumerate}
    \item % FIXME 24
\end{enumerate}

\subsubsection{Estoppel}

\paragraph{\emph{Hoddeson v. Koos Bros.}}

\begin{enumerate}
    \item % FIXME 28
\end{enumerate}

\newpage % FIXME remove

\subsection{Liability of Principal to Third Parties in Tort}

\subsubsection{Servant Versus Independent Contractor}

\paragraph{\emph{Humble Oil \& Refining Co. v. Martin}}

\begin{enumerate}
    \item % TODO 36
\end{enumerate}

\paragraph{\emph{Hoover v. Sun Oil Company}}

\begin{enumerate}
    \item % TODO 38
\end{enumerate}

\paragraph{\emph{Murphy v. Holiday Inns, Inc.}}

\begin{enumerate}
    \item % TODO 41
\end{enumerate}

\subsubsection{Tort Liability and Apparent Agency}

\paragraph{\emph{Miller v. McDonald's Corp.}}

\begin{enumerate}
    \item % TODO 47
\end{enumerate}

\subsubsection{Scope of Employment}

\paragraph{\emph{Ira S. Bushey \& Sons, Inc. v. United States}}

\begin{enumerate}
    \item % TODO 52
\end{enumerate}

% TODO restatement third:
    % 2.05
    % 2.06
    % 6.01-6.03
    % 2.04, 4.01, 4.02, 4.05(1), 6.10, 7.07, 8.01-05, 8.08-09
% rest second: 220
