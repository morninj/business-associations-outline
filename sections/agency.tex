\section{Agency}

\subsection{Who Is an Agent?}

\subsubsection{Definition of Agency: Restatement (Third) \S\S\ 1.01--1.03}

\begin{enumerate}
    \item ``Agency is the \textbf{fiduciary relationship} that arises when one 
    person (a `principal') manifests assent to another person (an `agent') 
    that the agent shall act on the principal's \textbf{behalf} and subject to 
    the principal's \textbf{control}, and the agent manifests assent or 
    otherwise \textbf{consents} so to act.'' \S\ 1.01.
    \item Other definitions of agency are not controlling. \S\ 1.02.
\end{enumerate}

\subsubsection{Actual and Apparent Authority: Restatement (Third) \S\S\ 
2.01--2.03}

\begin{enumerate}
    \item An agent acts with \textbf{actual authority} when he 
    \textbf{reasonably believes} that the principal wants him so to act. \S\ 
    2.01.
    \item The agent has actual authority to take actions that the principal 
    \textbf{designated or implied}, and to take actions that are 
    \textbf{necessary and incidental} to achieving the principal's objectives. 
    The reasonableness of the agent's interpretation of the principal's 
    manifestations and objectives can be inferred from context. \S\ 2.02.
    \item \textbf{Apparent authority} is the power of an agent or other actor 
    to affect the principal's legal relations with a \textbf{third party} when 
    the third party reasonably believes that the actor has authority to act on 
    behalf of the principal.
\end{enumerate}

\subsubsection{Presumption of Agency for Car Owners: \emph{Gorton v. Doty}}

There is a rebuttable presumption that the owner of a car is the principal and 
that the driver is his agent.

\begin{enumerate}
    \item Doty let Garst use her car to drive football players to a game. 
    Gorton was injured in an accident. Gorton sued Doty.\footnote{Casebook p. 
    1.}
    \item Was Garst an agent of Doty while he was driving the car?
    \item Definition of agency: see Restatement (Third) \S\ 1.01 (fiduciary 
    relationship, on the principal's behalf and subject to his control, and 
    the agent's consent).\footnote{See also casebook p. 2.}
    \item An agency relationship does not require a business matter, a 
    contract, or compensation. It only requires an agreement. (For instance, a 
    donative promise would establish agency, but it would not be enforceable 
    as a contract.)
    \item The court held that ownership of a car establishes a prima facie 
    case against the owner, establishing a presumption that the driver is the 
    owner's agent. Doty did not explicitly loan the car to Garst, so Garst was 
    her agent.
    \item Affirmed.
    \item Justice Budge, dissenting:
    \begin{enumerate}
        \item ``Agency means more than passive permission. It involves 
        request, instruction or command.''\footnote{Casebook p. 5.} (In other 
        words, there should not be a presumption that the owner of the car is 
        the principal and that the driver is his agent.)
    \end{enumerate}
\end{enumerate}

\subsubsection{Creditor as Principal: \emph{A. Gay Jenson Farms Co. v. 
Cargill, Inc.}}

A creditor that exercises enough control over a debtor's business can be 
liable as a principal for the debtor's acts.

\begin{enumerate}
    \item Farmers sold grain to Warren, a local grain elevator operator and 
    grain reseller. Warren entered into a financing agreement with Cargill. 
    After Warren defaulted, the farmers sued Warren and Cargill to recover 
    money that Warren owed them for grain.\footnote{Casebook p. 7--9.}
    \item Did Cargill's dealings with Warren cause it to become liable as a 
    principal on contracts Warren made with the farmers?\footnote{Casebook p. 
    9.}
    \item It's possible for a creditor to be liable as principals for the acts 
    of a debtors---for instance, by taking over management of the debtor's 
    business.\footnote{Casebook p. 10--11.}
    \item Cargill's relationship with Warren was more than that of a buyer and 
    supplier, and more than that of a financier. Rather, Cargill had a 
    ``paternalistic relationship'' with Warren that elevated their 
    relationship to that of a principal and agent.\footnote{Casebook p. 
    11--12.}
    \item This situation was different from ordinary bank financing because of 
    the level of control Cargill exercised.
\end{enumerate}

\subsection{Liability of Principal to Third Parties in Contract}

\subsubsection{The Agent's Authority}

\paragraph{\emph{Mill Street Church of Christ v. Hogan}}

\begin{enumerate}
    \item % FIXME 14
\end{enumerate}

\paragraph{\emph{Three-Seventy Leasing Corporation v. Ampex Corporation}}

\begin{enumerate}
    \item % FIXME 16
\end{enumerate}

\paragraph{\emph{Watteau v. Fenwick}}

\begin{enumerate}
    \item % TODO 20
\end{enumerate}

\subsubsection{Ratification}

\paragraph{\emph{Botticello v. Stefanovicz}}

\begin{enumerate}
    \item % TODO 24
\end{enumerate}

\subsubsection{Estoppel}

\paragraph{\emph{Hoddeson v. Koos Bros.}}

\begin{enumerate}
    \item % TODO 28
\end{enumerate}

\subsection{Liability of Principal to Third Parties in Tort}

\subsubsection{Servant Versus Independent Contractor}

\paragraph{\emph{Humble Oil \& Refining Co. v. Martin}}

\begin{enumerate}
    \item % TODO 36
\end{enumerate}

\paragraph{\emph{Hoover v. Sun Oil Company}}

\begin{enumerate}
    \item % TODO 38
\end{enumerate}

\paragraph{\emph{Murphy v. Holiday Inns, Inc.}}

\begin{enumerate}
    \item % TODO 41
\end{enumerate}

\subsubsection{Tort Liability and Apparent Agency}

\paragraph{\emph{Miller v. McDonald's Corp.}}

\begin{enumerate}
    \item % TODO 47
\end{enumerate}

\subsubsection{Scope of Employment}

\paragraph{\emph{Ira S. Bushey \& Sons, Inc. v. United States}}

\begin{enumerate}
    \item % TODO 52
\end{enumerate}

% TODO restatement third:
    % 2.05
    % 2.06
    % 6.01-6.03
    % 2.04, 4.01, 4.02, 4.05(1), 6.10, 7.07, 8.01-05, 8.08-09
% rest second: 220
