\section{Board of Directors} % FIXME rename

\subsection{Duty of Care}

\subsubsection{\emph{Kamin v. American Express Co.}}

% FIXME 308

\subsubsection{\emph{Smith v. Van Gorkom}}

% FIXME 312

\subsection{Duty of Loyalty}

\subsubsection{Directors and Managers}

\paragraph{Self Interest without Conflict of Interest: \emph{Bayer v. Beran}}

\begin{enumerate}
    \item A corporation contracted to advertise on a radio show where the wife 
    of the company's president and director performed.\footnote{Casebook p. 
    334.}
    \item Plaintiffs alleged a conflict of interest.
    \item Normally, this decision would fall under the business judgment rule, 
    but the wife's involvement raised a duty of loyalty issue, calling for 
    closer scrutiny.
    \item Held: no breach of duty because this was not intended to promote the 
    wife, and she did not gain more than the other performers. 
\end{enumerate}

\paragraph{Delaware---Strong Form Approach: \emph{Benihana of Tokyo, Inc. v. 
Benihana, Inc.}}
~\\\\
In Delaware, at least, the business judgment rule applies if the disinterested 
directors approve of the interested director transactions.

\begin{enumerate}
    \item Benihana sought financing for its restaurants. Its investment advisor 
    recommended an option which would dilute existing 
    shareholders.\footnote{Casebook p. 339. See also \emph{Understanding 
    Corporate Law} p. 259.}
    \item Abdo, one of the Benihana board members, contacted the advisor and 
    proposed adjusting the plan to involve BFC, a corporation in which Abdo had 
    a 30\% ownership stake.
    \item Aoki, the company's founder, objected because it would dilute his 
    ownership. But a committee of independent directors reviewed the plan and 
    concluded that the BFC plan was better.
    \item Plaintiffs argued that the transaction violated Delaware's interested 
    director statute.
    \item Held: Abdo had access to confidential information, but it was not used 
    to favor him, the negotiations were open, Benihana got much of what it 
    wanted, and Abdo did not deceive or dominate the board.
\end{enumerate}

\subsubsection{Corporate Opportunities}

\paragraph{\emph{Broz v. Cellular Information Systems, Inc.}}

\begin{enumerate}
    \item Broz owned a company that competed with CIS, and he served on CIS's 
    board.\footnote{Casebook p. 345. See also \emph{UCL} p. 274, 278.}
    \item Broz bought cellular licenses in Michigan. He contacted several CIS 
    officers and directors about the purchase, but was told CIS had no interest 
    because it had recently emerged from bankruptcy.
    \item PrimeCellular acquired CIS and sued Broz to recover the licenses, 
    claiming that Broz had usurped a corporate opportunity.
    \item Held: a corporate opportunity depends on whether the corporation can 
    exploit it, whether it is within the same line of business, whether there is 
    a corporate interest or expectancy, and whether the fiduciary's taking would 
    put him in opposition to his fiduciary duties. (None of these factors is 
    dispositive.) Since CIS could not acquire the licenses, Broz did not usurp a 
    corporate opportunity.
\end{enumerate}

\paragraph{\emph{In re eBay, Inc. Shareholders Litigation}}

% FIXME 350

\subsubsection{Dominant Shareholders}

\paragraph{\emph{Sinclair Oil Corp. v. Levien}}

% TODO 355

\paragraph{\emph{Zahn v. Transamerica Corporation}}

% TODO 359

\subsubsection{Ratification}

\paragraph{\emph{Fliegler v. Lawrence}}

% TODO 365

\subsection{The Obligation of Good Faith}

\subsubsection{Compensation: \emph{In re The Walt Disney Co. Derivative Litigation}}

% TODO 374-90

\subsubsection{Oversight: \emph{Stone v. Ritter}}

% TODO 390-98
