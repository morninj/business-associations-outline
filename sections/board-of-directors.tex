\section{Board of Directors} % FIXME rename

\subsection{Duty of Care}

\subsubsection{\emph{Kamin v. American Express Co.}}

% FIXME 308

\subsubsection{\emph{Smith v. Van Gorkom}}

% FIXME 312

\newpage % FIXME remove

\subsection{Duty of Loyalty}

\subsubsection{Directors and Managers}

\paragraph{Self Interest without Conflict of Interest: \emph{Bayer v. Beran}}

\begin{enumerate}
    \item A corporation contracted to advertise on a radio show where the wife 
    of the company's president and director performed.\footnote{Casebook p. 
    334.}
    \item Plaintiffs alleged a conflict of interest.
    \item Normally, this decision would fall under the business judgment rule, 
    but the wife's involvement raised a duty of loyalty issue, calling for 
    closer scrutiny.
    \item Held: no breach of duty because this was not intended to promote the 
    wife, and she did not gain more than the other performers. 
\end{enumerate}

\paragraph{Delaware---Strong Form Approach: \emph{Benihana of Tokyo, Inc. v. 
Benihana, Inc.}}
~\\\\
In Delaware, at least, the business judgment rule applies if the disinterested 
directors approve of the interested director transactions.

\begin{enumerate}
    \item Benihana sought financing for its restaurants. Its investment advisor 
    recommended an option which would dilute existing 
    shareholders.\footnote{Casebook p. 339. See also \emph{Understanding 
    Corporate Law} p. 259.}
    \item Abdo, one of the Benihana board members, contacted the advisor and 
    proposed adjusting the plan to involve BFC, a corporation in which Abdo had 
    a 30\% ownership stake.
    \item Aoki, the company's founder, objected because it would dilute his 
    ownership. But a committee of independent directors reviewed the plan and 
    concluded that the BFC plan was better.
    \item Plaintiffs argued that the transaction violated Delaware's interested 
    director statute.
    \item Held: Abdo had access to confidential information, but it was not used 
    to favor him, the negotiations were open, Benihana got much of what it 
    wanted, and Abdo did not deceive or dominate the board.
\end{enumerate}

\subsubsection{Corporate Opportunities}

\paragraph{\emph{Broz v. Cellular Information Systems, Inc.}}

\begin{enumerate}
    \item Broz owned a company that competed with CIS, and he served on CIS's 
    board.\footnote{Casebook p. 345. See also \emph{UCL} p. 274, 278.}
    \item Broz bought cellular licenses in Michigan. He contacted several CIS 
    officers and directors about the purchase, but was told CIS had no interest 
    because it had recently emerged from bankruptcy.
    \item PrimeCellular acquired CIS and sued Broz to recover the licenses, 
    claiming that Broz had usurped a corporate opportunity.
    \item Held: a corporate opportunity depends on whether the corporation can 
    exploit it, whether it is within the same line of business, whether there is 
    a corporate interest or expectancy, and whether the fiduciary's taking would 
    put him in opposition to his fiduciary duties. (None of these factors is 
    dispositive.) Since CIS could not acquire the licenses, Broz did not usurp a 
    corporate opportunity.
\end{enumerate}

\paragraph{\emph{In re eBay, Inc. Shareholders Litigation}}

% FIXME 350

\newpage % FIXME remove

\subsubsection{Dominant Shareholders}

\paragraph{\emph{Sinclair Oil Corp. v. Levien}}

\begin{enumerate}
    \item Sinclair controlled Sinven and International (and several other 
    subsidiaries). Sinclair owed a fiduciary duty to Sinven.\footnote{Casebook 
    p. 355.}
    \item Plaintiffs were 3\% shareholders in Sinven.
    \item When should courts apply the intrinsic fairness test (as opposed to 
    the business judgment rule) in parent-subsidiary relationships? Held: the 
    intrinsic fairness test applies when the parent controls the subsidiary, 
    when the decision involves self-dealing, and when and the decision 
    benefits the parent to the detriment and exclusion of the 
    subsidiary.\footnote{Casebook p. 356.}
    \item Plaintiffs brought three causes of action:
    \begin{enumerate}
        \item Sinven paid excessive dividends to Sinclair. Held: these 
        dividends were not self-dealing because all shareholders received 
        proportionate benefits; and they were consistent with the 
        business judgment rule because the plaintiffs could not show both that 
        the dividends resulted from improper motives and that they amounted to 
        waste.\footnote{Casebook p. 356--57.}
        \item Sinclair usurped corporate opportunities from Sinven. Held: no 
        opportunities came directly to Sinven, and Sinclair was under no 
        obligation to send opportunities to Sinven.\footnote{Casebook p. 
        357--58.}
        \item International breached its contract with Sinven by not making 
        timely payments and not selling minimum amounts. Sinclair breached its 
        duty by not requiring Sinven to enforce its contract. Held: Sinclair 
        did not prove that refusing to enforce the contract (through Sinven) 
        was fair.
    \end{enumerate}
\end{enumerate}

\paragraph{\emph{Zahn v. Transamerica Corporation}}

\begin{enumerate}
    \item Transamerica came to control Axton-Fisher. Zahn was a minority 
    shareholder of A-F.
    \item A-F had bought \$6 million worth of tobacco. The value shot up to 
    \$20 million. TA exercised its option to buy out Zahn's class of shares, 
    and then it liquidated the company, keeping the tobacco for itself.
    \item Zahn argued that TA was not allowed to direct the board of A-F to 
    act in its benefit, to the exclusion of its other shareholders. The court 
    agreed, instruction TA to share its windfall with the other shareholders.
\end{enumerate}

\subsubsection{Ratification}

\paragraph{\emph{Fliegler v. Lawrence}}

% TODO 365

\newpage % TODO remove

\subsection{The Obligation of Good Faith}

\subsubsection{Compensation: \emph{In re The Walt Disney Co. Derivative Litigation}}

\begin{enumerate}
    \item Disney hired Ovitz as its President. It fired him after a year, 
    resulting in a \$130 million severance payout.\footnote{Casebook p. 374.}
    \item Ovitz's employment agreement and election as 
    President:\footnote{Casebook p. 379--386.}
    \begin{enumerate}
        \item Did Disney act with due care? Yes.
        \item Did Disney act in good faith? Yes.
        \begin{enumerate}
            \item \textbf{Good faith} is a spectrum:\footnote{Casebook p. 
            384--96.}
            \begin{enumerate}
                \item \textbf{Subjective bad faith}: intent to do harm.
                \item \textbf{``Intentional dereliction of duty, a conscious 
                disregard for one's responsibilities''}: e.g., acts with a 
                purpose other than advancing the corporation's interests. 
                Cannot be exculpated. (This is a non-exclusive definition.)
                \item \textbf{Lack of due care}: indistinguishable from gross 
                negligence.
            \end{enumerate}
        \end{enumerate}
    \end{enumerate}
    \item Severance payout:\footnote{Casebook p. 386--388.}
    \begin{enumerate}
        \item Eisner has the authority to unilaterally terminate Ovitz's 
        employment.\footnote{Casebook p. 387.}
        \item Disney was correct to determine that Eisner could not be 
        terminated for cause.
    \end{enumerate}
    \item Waste: a meritless claim because Disney was contractually obligated 
    to make the severance payments.\footnote{Casebook p. 388-89.}
\end{enumerate}

\subsubsection{Oversight: \emph{Stone v. Ritter}}

\begin{enumerate}
    \item AmSouth paid fines to the federal government for failure to 
    implement adequate anti--money laundering programs.\footnote{Casebook p. 
    390--93.}
    \item When is a director personally liable for failing to act in good 
    faith in exercizing oversight? \emph{Caremark} established the criteria:
    \begin{enumerate}
        \item ``(a) the directors utterly failed to implement any reporting or 
        information systems or controls; or
        \item (b) having implemented such a system or controls, consciously 
        failed to monitor or oversee its operations thus disabling themselves 
        from being informed of risks or problems requiring their 
        attention.''\footnote{Casebook p. 395.}
    \end{enumerate}
    \item Held: AmSouth had implemented an oversight program as recommended by 
    outside consultants, and it relied on periodic reports from the program. 
    It did not fail to exercise adequate oversight.\footnote{Casebook p. 397.}
\end{enumerate}
