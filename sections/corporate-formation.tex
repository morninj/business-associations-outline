a\section{Corporate Formation}

\subsection{The Corporate Entity and Limited Liability}

\subsubsection{\emph{Walkovsky v. Carlton}}

\begin{enumerate}
    \item Carlton organized 10 taxi corporations, each owning two cabs and 
    carrying the minimum allowed liability insurance. The plaintiff argued 
    that they operated as a single enterprise, and that Carlton is personally 
    liable because he structured his business to defraud the 
    public.\footnote{Casebook p. 176.}
    \item Courts can pierce the corporate veil to prevent fraud or achieve 
    equity.
    \item When anyone (either a person or a corporation) uses a corporation 
    for his personal ends, he can be liable under respondeat superior for the 
    corporation's acts.\footnote{Casebook p. 176--77.}
    \item Here, there was no evidence that Carlton was conducting business in 
    his personal capacity. So, Carlton is not personally liable under an 
    agency theory.
    \item Moreover, there was no fraud. It is not fraudulent for a taxi 
    company to take out the minimum required insurance, nor is it illicit to 
    organize its enterprise as many separate corporations.\footnote{Casebook 
    p. 178.}
    \item Judge Keating, dissenting: these corporations were intentionally 
    undercapitalized in order to avoid responsibility. ``What I would merely 
    hold is that a participating shareholder of a corporation vested with a 
    public interest, organized with capital insufficient to meet liabilities 
    which are certain to arise in the ordinary course of the corporation's 
    business, may be held personally responsible for such 
    liabilities.''\footnote{Casebook p. 178--80.}
\end{enumerate}

\subsubsection{\emph{Sea-Land Services, Inc. v. Pepper Source}}

\begin{enumerate}
    \item PS stiffed Sea-Land on a shipping bill. Marchese controlled PS and 
    four other businesses. Sea-Land alleged that the corporations were alter 
    egos of each other and of Marchese himself, because he created and used 
    them for his personal benefit.\footnote{Casebook p. 182.}
    \item The \emph{Van Dorn} test for piercing the corporate veil has two 
    requirements:\footnote{Casebook p. 182--83.}
    \begin{enumerate}
        \item ``[T]here must be such \textbf{unity of interest and ownership} 
        that the separate personalities of the corporation and the individual 
        [or other corporation] no longer exist.''
        \item ``[C]ircumstances must be such that adherence to the fiction of 
        separate corporate existence would \textbf{sanction a fraud or promote 
        injustice}.''
    \end{enumerate}
    \item First requirement: yes, there was unity of interest and 
    ownership.\footnote{Casebook p. 183--184.}
    \item Second requirement: Sea-Land did not argue fraud, but it did argue 
    that adherence to Marchese's scheme would ``promote injustice.'' The court 
    here held that ``promoting injustice'' requires a wrong ``beyond a 
    creditor's inability to collect'' (e.g., it would undermine the common 
    sense rules of adverse possession). Sea-Land failed to give evidence of 
    such a wrong.\footnote{Casebook p. 184--86.}
    \item Summary judgment in favor of Sea-Land was reversed.
\end{enumerate}

\subsubsection{\emph{In re Silicone Gel Breast Implants Products Liability 
Litigation}}

\begin{enumerate}
    \item Bristol was the sole shareholder of MEC, a breast implant 
    manufacturer. Bristol sought summary judgment in a products liability 
    case.\footnote{Casebook p. 191.}
    \item Bristol retained tight control of MEC's operations after buying 
    it.\footnote{Casebook p. 191--94.}
    \item Plaintiffs sought to pierce the corporate veil and hold Bristol 
    liable for MEC's actions. (They also argued that Bristol was directly 
    liable on other counts.)\footnote{Casebook p. 194.}
    \item ``~.~.~.~when a corporation is so controlled as to be the 
    \textbf{alter ego} or \textbf{mere instrumentality} of its stockholder, 
    the corporate form may be disregarded in the interests of 
    justice.''\footnote{Casebook p. 194.}
    \item Several factors can determine whether a corporation is a mere 
    instrumentality or an alter ego---see p. 195. A showing of fraud is not 
    necessary.\footnote{Casebook p. 196.}
    \item Held: a jury could find that MEC was an alter ego or mere 
    instrumentality. Summary judgment denied for Bristol.
\end{enumerate}

\newpage % TODO remove

\subsection{The Role and Purposes of Corporations}

% TODO 251-67

% TODO 
% Model Business Corporation Act: Sections 2.01-2.06, 6.21, 6.22(b)&(c), 6.40(c), 8.33(a)
% Minute Book: (Just skim) Articles, Bylaws, and Written Consent in Lieu of Organizational Meeting
