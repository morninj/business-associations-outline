\section{Corporate Litigation}

% TODO ac slides 3/20:
%     direct vs. derivative claims
%         how to decide which type to bring? apply the tooley test
%         it's pretty hard to bring a derivative claim. shareholders almost always lose.
%     grimes v. donald
%         excessive compensation claim (beneficial severance -- similar to disney)
%         when you make a demand on the board, you're conceding that demand is required
%             boards usually don't pursue demands. so plaintiffs usually don't bring them; instead, they file derivative suits
%             so plaintiffs usually try to show that demand would be futile -- 
%                 under  the Aronson test from grimes
% 
%     ** when is demand excused? aronson test. see ac slides.

\subsection{Indemnification and Insurance}

\begin{enumerate}
    \item Several situations can give rise to liability---e.g., claims by 
    third persons, or derivative suits.\footnote{Casebook p. 503.}
    \item Damages can be large in relation to the wealth of officers.
    \item Should corporations be allowed to indemnify officers or directors?
    \item Officers should sometimes worry if reimbursement is left to the 
    discretion of the board. What is the board turns hostile?
\end{enumerate}

\subsubsection{\emph{Waltuch v. Conticommodity Servs., Inc.}}

\begin{enumerate}
    \item Waltuch lost money in his capacity as a silver trader. He paid \$2.2 
    million to defend himself against civil suits and an enforcement action by 
    the CFTC. Here, he sought indemnity from Conti.
    \item First claim:\footnote{Casebook p. 505--09.}
    \begin{enumerate}
        \item The statute (Delaware General Corporation Law \S\ 145) prevented 
        corporations from indemnifying a person when the person did not act 
        in good faith.
        \item Held: Conti and Waltuch could not contract around this 
        limitation. Since Waltuch did not act in good faith, the corporation 
        does not have to indemnify him.
    \end{enumerate}
    \item Second claim:\footnote{Casebook p. 509--11.}
    \begin{enumerate}
        \item The civil suits settled. \S\ 145(c) requires corporations to 
        indemnify officers and directors as defendants who are ``successful'' 
        on certain claims.
        \item Held: Conti's settlement, under which Waltuch was not required 
        to make any payments, was a ``success'' within the meaning of the 
        statute.
        \item Conti must indemnify Waltuch for defense expenses in the civil 
        cases (\$1.2 million).
    \end{enumerate}
\end{enumerate}
