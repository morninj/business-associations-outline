\section{Limited Liability Companies}

\subsection{Limited Partnerships}

\subsubsection{\emph{Holzmann v. De Escamilla}}

\begin{enumerate}
    \item % FIXME 166-68
\end{enumerate}

\subsection{LLC Overview}

\begin{enumerate}
    \item % FIXME 268
    \item Management: either \textbf{member-managed} (akin to a partnership) 
    or \textbf{manager-managed} (owners delegate management to others). % TODO see cable slides
    \item By default, every member is an agent of the LLC (but this, like 
    everything else, can be changed in the operating agreement). If there is a 
    dispute, a majority vote will decide the outcome. If there are only two 
    members who cannot agree, you resort to the status quo (same as 
    partnership law; see \emph{Stroud}).
    \item All members have apparent authority to act on the LLC's behalf.
\end{enumerate}

\subsection{Operating Agreement}

\subsubsection{Freedom of Contract: \emph{Elf Atochem North America, Inc. v. 
Jaffari}}

Courts give broad deference to operating agreements 

\begin{enumerate}
    \item Jaffari operated Malek, Inc. Malek, Inc. and Elf, Inc. decided to 
    undertake a joint venture in the form of Malek, LLC, with Jaffari as the 
    manager and CEO. Malek, Inc. contributed its IP for 70\% interest, and Elf 
    contributed \$1 million for 30\% interest and exclusive distribution 
    rights.\footnote{Casebook p. 274--75.}
    \item The agreement provided an arbitration clause and a forum selection 
    clause, setting California as the forum for all dispute resolution.
    \item Later, Elf sued Jaffari and Malek, LLC in Delaware for several tort 
    and contract claims.
    \item Jaffari wanted to enforce the California arbitration clause. The 
    court upheld the contract.
    \item Derivative vs. direct claims (more to come later in corporate law):
    \begin{enumerate}
        \item Elf characterized is claims as derivative, rather than direct.
        \item \textbf{Derivative suit}: Some lawsuits must be brought in the 
        name of the corporation---e.g., when an officer breaches his fiduciary 
        duty, the corporation itself sues the officer. This allows for 
        collective decisionmaking (i.e., it prevents a single shareholder from 
        wasting corporate assets by bringing a frivolous claim against an 
        officer).
        \item \textbf{Direct suit}: a shareholder brings it in his individual 
        capacity.
    \end{enumerate}
    \item (To help mitigate this problem, Elf could have set up a three-person 
    management structure with one of its own as a manager.)
\end{enumerate}

\subsubsection{Veto Power: \emph{Fisk Ventures, LLC v. Segal}}

\begin{enumerate}
    \item Segal developed a drug. He formed Genitrix, LLC with three classes of 
    members:\footnote{Casebook p. 280 ff.}
    \begin{enumerate}
        \item Class A: Segal and his friends. Contributed IP and \$1 million cash.
        \item Class B: all somehow related to Fisk. Contributed about \$1.4 
        million cash.
        \item Class C: truly passive. Donated \$1 million cash.
    \end{enumerate}
    \item Managers: two from class A, 3 from class B, with Segal as CEO. (So, 
    this is sort of a board of managers.)
    \item The rights of the three classes of members are defined in the 
    operating agreement. (See \S\ 1.1 of the sample complicated operating 
    agreement.)
    \item ``Board of managers'' (LLC; a contractual fiction which does not 
    appear in the LLC statute) vs. board of directors (corp.):\footnote{See 
    Cable slides 2/4/2013. Important for choice of entity.}
    \begin{enumerate}
        \item Board of managers: default rule of equal management rights and 
        actual authority (ULLCA \S\S\ 301 and 404(b)). Apparent authority 
        (ULLCA \S\ 301). No default meeting or procedural requirements.
        default rule of equal management rights and 
        actual authority (ULLCA \S\S\ 301 and 404(b)). Apparent authority 
        (ULLCA \S\ 301). No default meeting or procedural requirements.
        \item Board of directors: default oversight function. No management 
        rights or apparent authority. Strong meeting and procedural 
        requirements. Lots of developed caselaw and custom.
    \end{enumerate}
    \item Here: new financing required 75\% of managers to approve. So, we 
    have another deadlock problem. The company was ultimately unable to reach 
    an agreement with the class B members for additional investment or 
    necessary concessions.
    \item (How could you prevent deadlock in this scenario? You could 
    implement a structure in which the parties agree to a trusted 
    tiebreaker---an ``independent manager.'')
\end{enumerate}

\subsection{Piercing the ``LLC'' Veil}

\subsubsection{\emph{Kaycee Land and Livestock v. Flahive}}

% FIXME 287 ff.

\subsection{Fiduciary Obligation}

\subsubsection{\emph{McConnell v. Hunt Sports Enterprises}}

% FIXME 292-97

% TODO ULLCA 201-03, 301, 303, 404
% minute book operating agreement
