\section{Mergers and Acquisitions}

\subsection{Takeovers}

\subsubsection{Greenmail}

\begin{enumerate}
    \item Scare the management of a company and get paid to go away.
    % TODO expand--p. 748
\end{enumerate}

\subsubsection{Takeover Defenses: \emph{Unocal Corp. v. Mesa Petrol. Co.}}

\begin{enumerate}
    \item Mesa made a ``two-tier front-loaded cash tender offer'' for Unocal's 
    shares. (See pp, 749--51.)
    \item Facts (all in 1985):\footnote{Casebook p. 752--53.}
    \begin{enumerate}
        \item April 8: Mesa made its offer---\$54/share. (Unocal was trading at 
        a significantly lower level---around \$40. Why? Because of all of its 
        drilling operations, which were expensive, and they weren't producing a 
        positive return. So: maybe there was mismanagement. Also, the stock 
        market price is the price for a single stock. But Mesa's offer is for 
        the entire company. Controlling shares are more valuable.)
        \begin{enumerate}
            \item It would have been rational for most of the shareholders to 
            accept \$54 in Mesa's offer, rather than wait for the second stage, 
            when they would get less. (It would only be rational for them not to 
            tender if they, acting alone, had enough shares to prevent the 
            transaction.)
        \end{enumerate}
        \item April 13: Unocal's board met and determined that Mesa's offer was 
        inadequate. Actual value according to Goldman Sachs: \$60.
        \item April 15: Unocal's board met again and decided to take a 
        \textbf{defensive measure}: it approved a self-tender exchange offer 
        that would allow it to buy other shares from all shareholders except 
        Mesa at a price higher than Mesa's offer (\$72/share).
        \begin{enumerate}
            \item The lower court enjoined this measure. It held that the 
            exclusion of Mesa requires (1) a valid business purpose and (2) that 
            the transaction is fair to everyone, including Mesa's---so, in other 
            words, Unocal could not exclude Mesa.\footnote{Casebook p. 752.}
            \item This would dilute the value of the 13\% that Mesa already 
            owns, because the company now has more debt that it incurred to pay 
            the \$72. (The shareholders other than Mesa are not diluted because 
            they get the \$72. So only Mesa gets injured.) Mesa would take a big 
            financial hit if it bought the company for \$54.
        \end{enumerate}
        \item April 17: Unocal made its selective exchange offer and Mesa sued.
        \item April 29: Mesa won a temporary restraining order preventing the 
        exchange offer unless it included Mesa.
    \end{enumerate}
    \item The court is \textbf{concerned about entrenchment} because the board 
    is taking action to prevent the shareholders from deciding on Mesa's offer. 
    This is sort of \textbf{intermediate scrutiny}, somewhere between 
    \textbf{entire fairness} (i.e., strict scrutiny) and the \textbf{business 
    judgment rule} (i.e., a large degree of deference).
    \item \textbf{The two-part Unocal test}:
    \begin{enumerate}
        \item \textbf{Motive}: did the board identify danger to corporate policy 
        or effectiveness (through good faith and reasonable investigation, 
        preferably by independent directors)? Potentially valid motivations:
        \begin{enumerate}
            \item Amount or form of consideration (the most important one).
            \item Timing of offer.
            \item Risk that deal won't get done (such as financing or regulatory 
            approval).
            \item Maybe non-shareholder effects (but \emph{Revlon}, below, 
            whittles this one way down, so it rarely comes up).
            \item Long-term strategic vision (e.g., Time magazine \& editorial 
            integrity).
        \end{enumerate}
        \item \textbf{Proportionality}: was the defensive measure reasonable in 
        relation threat posed?
    \end{enumerate}
    \item How did Unocal's defensive exclusion of Mesa fare under this test?
    \begin{enumerate}
        \item \textbf{Motive}: the back-end part of Mesa's offer (the junk 
        bonds) was inadequate. So, Unocal had a valid motive.
        \item \textbf{Proportionality}: Unocal's self-tender offer simply would 
        not have worked if it allowed Mesa to participate. If Mesa had been 
        included, then the self-tender offer would have just subsidized Mesa's 
        threat.
        \item (No need for the action to be narrowly tailored.)
        \item (Some courts are more likely to find actions to be proportional if 
        the one doing the takeover (here, Mesa) has another way to achieve the 
        takeover.)
    \end{enumerate}
    \item Unocal's self-tender offer passed the two-part test, so the board's 
    action is reviewed under the business judgment rule, and will only be 
    invalidated if the board breached its fiduciary duty.
\end{enumerate}

\subsubsection{SEC's Reaction to \emph{Unocal} and Poison Pills}

% TODO 759--61
% and see AC slides on poison pills from 3/20; they work like the rights plan in 
% Revlon
%
% poison pills usually pass the unocal test when they're used to drive up bids 
% for the company; but they would fail if you just used them perpetually to 
% protect your job

\subsubsection{Deal Protection Devices: \emph{Revlon, Inc. v. MacAndrews \& 
Forbes Holdings}}

% When a board sews up a deal too tightly to the exclusion of other bidders, % TODO 

\begin{enumerate}
    \item Target: Revlon.
    \item Hostile bidder: Pantry Pride (PP).
    \item White knight: Forstmann Little (FL; a buyout specialist).
    \item Facts:
    \begin{enumerate}
        \item % TODO see AC slides 3/18
        \item Note holders don't like the deal with FL because the deal would 
        waive their restrictive covenants.
        \item Then: FL somehow supports the notes (to quiet the note holders). 
        FL agrees to buy Revlon for \$57.25/share. And FL gets deal protection 
        devices from Revlon: an \textbf{asset lock-up}, a \textbf{no-shop provision}, and a 
        \textbf{termination fee}. That's the structure of Revlon's proposed deal 
        with FL.
    \end{enumerate}
    \item First takeover defense---the \textbf{note purchase rights plan}:
    \begin{enumerate}
        \item Revlon distributes ``rights'' to all of its shareholders. What 
        does the right give them?
        \begin{enumerate}
            \item Allows them to exchange stock for notes.
            \item The notes are IOUs from Revlon.
            \item They're very favorable to shareholders---about \$65 in value 
            for each note.
            \item These rights are not available to PP.
            \item If anyone acquires more than 20\% of Revlon stock, then all 
            Revlon stockholders' stock are converted to notes.
            \item Revlon can redeem the rights plan anytime before anyone 
            acquires 20\%.
        \end{enumerate}
        \item Bottom line: this drives up the acquisition cost of PP because it 
        would dilute the value of Revlon if PP acquired it by converting stocks 
        to notes (and the notes would create big debt for Revlon); so it gives the 
        Board the ability to choose who buys the company.
    \end{enumerate}
    \item Second takeover defense---the \textbf{poison debt}:
    \begin{enumerate}
        \item Designed to prevent the type of corporate-level actions that a 
        company would want to take when it acquires another company. It would 
        make it a major hassle to take over Revlon.
        \item % TODO see AC slides
        \item Bottom line: makes it difficult for PP to acquire Revlon. 
    \end{enumerate}
    \item How do the rights plan and poison debt fare under the two-part 
    \emph{Unocal} test?---Just fine:
    \begin{enumerate}
        \item \textbf{Motive}: price inadequacy is a valid motive.
        \item \textbf{Proportionality}: the rights plan and poison debt were 
        proportional because they effectively bid up the price of Revlon's 
        shares.
    \end{enumerate}
    \item % TODO continue on thurs 3/20
\end{enumerate}

\subsubsection{What triggers \emph{Unocal}?}

\begin{enumerate}
    \item Takeover defenses.
    \item (So, \emph{Unocal} is triggered if the target uses a poison pill. A 
    court would then evaluate the poison pill under \emph{Unocal}.)
    \item (On an exam, no need explain \emph{how to} apply \emph{Revlon}---we 
    don't know enough yet---but do indicate \emph{whether} it applies.)
\end{enumerate}

\subsubsection{What triggers \emph{Revlon}?}

I.e., this is when the board's role becomes that of an auctioneer (i.e., get  
the best price for the company).

\begin{enumerate}
    \item Self-initiated sale of company for cash.
    \item Self-initiated stock-for-stock deal (i.e., shareholders of the target 
    exchange their shares for shares of the acquirer) \textbf{resulting in a 
    controlling shareholder} (or controlling shareholder group) of the combined 
    entity.
    \item (Includes white knight deals matching these patterns.)
\end{enumerate}

What does \emph{not} trigger \emph{Revlon}:

\begin{enumerate}
    \item Board simply refuses an unsolicited offer.
    \item Stock-for-stock deal when the control of the new combined entity 
    remains in unaffiliated hands (i.e., \textbf{if control resides in the 
    market}, rather than a single controlling shareholder).
\end{enumerate}

\subsubsection{Limiting \emph{Revlon}: \emph{Paramount Commc'ns, Inc. v. Time, 
Inc.}}

\begin{enumerate}
    \item Facts:\footnote{Casebook p. 772--76.}
    \begin{enumerate}
        \item Time's board favored a merger with Warner.
        \item Time's and Warner's boards approved the merger.
        \item Then, Paramount offered to buy Time for \$175/share. Time rejected 
        the offer.
        \item As a defensive measure, Time adjusted its deal with Warner to be 
        an outright acquisition of Warner.
        \item Paramount raised its offer to \$200/share. Time again rejected.
    \end{enumerate}
    \item \emph{Revlon} claim:\footnote{Casebook p. 776--78.}
    \begin{enumerate}
        \item \emph{Revlon} test: when the dissolution or breakup of a company 
        is inevitable, the board must maximize stockholder value by securing the 
        highest price available.
        \item Paramount argued that Time's deal with Warner effectively put Time 
        up for sale, triggering \emph{Revlon} duties---i.e., requiring Time's 
        board to maximize stockholder value by securing the highest possible 
        price for the company.
        \item Held: the Time-Warner negotiations did not make the dissolution or 
        breakup of Time inevitable. Since the deal did not constitute an 
        abandonment of the corporation's continued existence, it did not trigger 
        \emph{Revlon} duties.
    \end{enumerate}
    \item \emph{Unocal} claim:\footnote{Casebook p. 778--81.}
    \begin{enumerate}
        \item \emph{Unocal} test: the business judgment rule applies to a 
        board's defensive measure if (1) a \textbf{threat} to the corporation 
        existed and (2) the defensive tactic was \textbf{reasonable} at 
        \textbf{proportional}.
        \item All parties agreed that \emph{Unocal} applied to Time's second 
        deal with Warner because the deal was a defensive measure.
        \item Paramount argued that its offer was not a threat, so the Time 
        board's defensive measure failed the first part of the \emph{Unocal} 
        test.
        \item Held: a broad range of threats can satisfy the first prong of the 
        \emph{Unocal} test, including an attempt like Paramount's to upset or 
        confuse a shareholder vote.
        \item Also held: Time's action was reasonable. Boards can set their own 
        time frames for achieving corporate goals.
    \end{enumerate}
\end{enumerate}

\subsubsection{\emph{Paramount Commc'ns, Inc. v. Time, Inc.}}

\begin{enumerate}
    \item % TODO 781-98
\end{enumerate}
