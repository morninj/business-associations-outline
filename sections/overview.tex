\section{Overview}

\subsection{Agency}

\begin{enumerate}
    \item When is someone responsible for the actions of another? When would 
    the principal feel indignant?
    \item \textbf{Three actors}: principal, agent, third party. The third 
    party is bringing the claim.
    \item Two types of cases: \textbf{contract} and \textbf{tort}.
    \item Is there an \textbf{agency relationship}? Elements:\footnote{See 
    Restatement (Third) \S\ 1.01. This step of the analysis applies to both 
    contract and tort cases.}
    \begin{enumerate}
        % TODO review restatement examples
        \item The principal's \textbf{manifestation of assent}. (This can 
        exist even if the principal explicitly denies an agency relationship. 
        The terminology of the agreement is not controlling.\footnote{Casebook 
        p. See, e.g., \emph{Cargill}, where the lawyers disclaimed an agency 
        relationship in the initial agreement.})
        \item Agent acting on the principal's \textbf{behalf}. (Do the agent's 
        actions directly benefit the principal?)
        \item Agent subject to the principal's \textbf{control}.
        \item Agent \textbf{consents}.
    \end{enumerate}
    \item If this is a \textbf{breach of contract} case:
    \begin{enumerate}
        \item Did the agent have \textbf{authority} to act? Types of authority 
        (which often overlap):\footnote{This step applies \emph{only} to 
        contract cases.}
        \begin{enumerate}
            \item \textbf{Actual}: can be (1) express or (2) implied. \S\S\ 
            2.01--2.02.
            \item \textbf{Apparent}: did the third party reasonably believe 
            the agent had authority, and was that belief \emph{traceable} to 
            the principal's manifestations? \S\ 2.03.
            \item (These often overlap---e.g., \emph{Mill Street Church}, p.  
            \pageref{par:mill}.)
        \end{enumerate}
        \item Do any \textbf{exceptions} apply?\footnote{These are substitutes 
        for authority.}
        \begin{enumerate}
            \item \textbf{Ratification}: requires the principal's (1) 
            \textbf{intent to ratify} and (2) \textbf{full knowledge} of 
            material circumstances. \emph{Botticello v. Stefanovicz}, p.  
            \pageref{par:botticello}. See also Restatement (Third) \S\S\ 4.01 
            (definition), 4.02 (effect), and 4.05(1) (timing).
            \item \textbf{Estoppel}: % TODO hoddeson
            \item \textbf{Undisclosed principal}: % TODO watteau
        \end{enumerate}
    \end{enumerate}
    \item If this is a \textbf{tort} case:
    \begin{enumerate}
        \item Was the person an employee or independent contractor? % TODO 
        % TODO  an employer is liable for the employee's acts
        %when the employee is acting within the scope of employment
        \item If the person was neither an employee nor an independent 
        contractor, was there \textbf{apparent agency}? % TODO see mcdonald's
    \end{enumerate}
    \item Setting the \textbf{purpose and conditions} of activity can 
    establish an agency relationship. \emph{Gorton v. Doty}, p. 
    \pageref{subsub:gorton}.
    \item A creditor that exercises enough control over a debtor's business 
    can be liable as a principal for the debtor's acts. \emph{A. Gay Jenson 
    Farms Co. v. Cargill}, p. \pageref{subsub:cargill}.
    \item What would be the \textbf{consequences} of finding an agency 
    relationship?
    \begin{enumerate}
        \item We wouldn't want debtors to always be the agents of their 
        creditors, because that would effectively kill financing. But there 
        are cases where the creditor exerts enough control for an agency 
        relationship to exist. \emph{A. Gay Jenson Farms Co. v. Cargill}, p. 
        \pageref{subsub:cargill}.
    \end{enumerate}
    \item \textbf{Strategies for avoiding liability}:
    \begin{enumerate}
        \item (First, define the \textbf{risks and rewards}.)
        \item Get \textbf{insurance}. \emph{Gorton v. Doty}, p.
        \pageref{subsub:gorton}.
        \item \textbf{Legal structuring}: structure the relationship to negate 
        one of the four elements.
        \begin{enumerate}
            \item Specify the nature of the relationship, ideally in writing.
            \item Delegate control to another principal, or give up control 
            entirely.
        \end{enumerate}
        \item \textbf{Security/collateral}. E.g., a right of first refusal. 
        \emph{A. Gay Jenson Farms Co. v. Cargill}, p.  
        \pageref{subsub:cargill}. \emph{Cargill}.
        \item \textbf{Monitoring}. E.g., accounting audits.
        \item \textbf{Operational control}. E.g., a sign-off requirement for 
        major decisions. \emph{Cargill} again.
        \item \textbf{Due diligence}. E.g., verify the type of property 
        interest that the seller is selling. \emph{Botticello v. Stefanovicz}, 
        p. \pageref{par:botticello}.
        \item \textbf{Recording/registration systems}. E.g., check public land 
        records. \emph{Botticello}.
    \end{enumerate}
\end{enumerate}

% TODO 
