\section{Overview}

\subsection{Agency}

\begin{enumerate}
    \item Ask: is the principal liable to a third party for another's actions?
    \item \textbf{Three actors}: principal, agent, third party. The third 
    party is bringing the claim.
    \item Two types of cases: \textbf{contract} and \textbf{tort}.
    \item Is there an \textbf{agency relationship}? Elements:\footnote{See 
    Restatement (Third) \S\ 1.01. This step of the analysis applies to both 
    contract and tort cases.}
    \begin{enumerate}
        \item The principal's \textbf{manifestation of assent}. (This can 
        exist even if the principal explicitly denies an agency relationship. 
        The terminology of the agreement is not controlling.\footnote{Casebook 
        p. See, e.g., \emph{Cargill}, where the lawyers disclaimed an agency 
        relationship in the initial agreement.})
        \item Agent acting on the principal's \textbf{behalf}. (Do the agent's 
        actions directly benefit the principal?)
        \item Agent subject to the principal's \textbf{control}.
        \item Agent \textbf{consents}.
        \item Examples:
        \begin{enumerate}
            \item Setting the \textbf{purpose and conditions} of activity can 
            establish an agency relationship. \emph{Gorton v. Doty}, p.  
            \pageref{subsub:gorton}.
            \item A creditor that exercises enough control over a debtor's 
            business can be liable as a principal for the debtor's acts. 
            \emph{A. Gay Jenson Farms Co. v. Cargill}, p. 
            \pageref{subsub:cargill}.
        \end{enumerate}
    \end{enumerate}
    \item If this is a \textbf{breach of contract} case:
    \begin{enumerate}
        \item Did the agent have \textbf{authority} to act? Types of authority 
        (which often overlap):\footnote{This step applies \emph{only} to 
        contract cases.}
        \begin{enumerate}
            \item \textbf{Actual}: can be (1) express or (2) implied. \S\S\ 
            2.01--2.02.
            \item \textbf{Apparent}: did the third party reasonably believe 
            the agent had authority, and was that belief \emph{traceable} to 
            the principal's manifestations? \S\ 2.03.
            \item (These often overlap---e.g., \emph{Mill Street Church}, p.  
            \pageref{par:mill}.)
        \end{enumerate}
        \item Do any \textbf{exceptions} apply?\footnote{These are substitutes 
        for authority.}
        \begin{enumerate}
            \item \textbf{Ratification}: requires the principal's (1) 
            \textbf{intent to ratify} and (2) \textbf{full knowledge} of 
            material circumstances. \emph{Botticello v. Stefanovicz}, p.  
            \pageref{par:botticello}. See also Restatement (Third) \S\S\ 4.01 
            (definition), 4.02 (effect), and 4.05(1) (timing).
            \item \textbf{Estoppel}: a third party can assert estoppel when 
            he (1) \textbf{detrimentally relies} on an impostor agent and (2) 
            the principal \textbf{intentionally or carelessly} caused the 
            belief or the principal \textbf{was on notice} and failed to take 
            reasonable steps. Restatement (Third) \S\ 2.05 and 
            \emph{Hoddeson}, p. \pageref{par:hodd}.
            Estoppel only creates liability for the 
            principal (i.e., the principal can't assert estoppel against the 
            third party).
            \item \textbf{Undisclosed principal}: if the agent lacks 
            authority (either actual or apparent), the undisclosed principal 
            can still be liable for acts that are ``within the authority 
            usually confided'' to agents. See \emph{Watteau}, p. 
            \pageref{par:watteau}. Requires an underlying agency 
            relationship.
        \end{enumerate}
    \end{enumerate}
    \item If this is a \textbf{tort} case:
    \begin{enumerate}
        \item \textbf{Respondeat superior}: employers are liable for torts 
        their \textbf{employees} commit while within the \textbf{scope of 
        employement}. Restatement (Third) \S\ 2.04.
        \begin{enumerate}
            \item \textbf{Definition of employee}: is an agent (under 
            Restatement \S\ 1), and the principal controls or has the right to 
            control the \textbf{manner and means} of the work. Restatement 
            (Third) \S\ 7.07(3).
            \begin{enumerate}
                \item Servant/employee: principal is liable for torts if the 
                employee was acting within the scope of employment.
                \item Independent contractor (agent-type): P is liable for 
                contracts but not torts.
                \item Independent contractor (non-agent): P is not liable.
            \end{enumerate}
            \item \textbf{Definition of scope of employement}: doesn't include 
            independent courses of conduct not intended to serve the 
            employer's purposes. Restatement (Third) \S\ 7.07.  
            \end{enumerate}
        \item \textbf{Exceptions} (under which P is liable):
        \begin{enumerate}
            \item Engaging an incompetent contractor.
            \item Nondelegable duty.
        \end{enumerate}
        \item \textbf{Employees vs. independent contractors} (Restatement 
        (Second) \S\ 220(2)):
        \begin{enumerate}
            \item P \emph{may} exercise control over details of A’s work
            \item A does not engage in distinct business
            \item Type of work typical of supervised employee, not unsupervised specialist
            \item Job requires low skill level
            \item P supplies tools and dictates place of work
            \item A employed for a long term
            \item A is compensated by time and not by job
            \item Work is a regular part of P’s business
            \item Intent to create employer/employee relationship
            \item P is in business
            \item See gas station cases (\emph{Humble} and \emph{Sun Oil}) for 
            similar facts reaching different conclusions on the 
            employee-vs.-independent-contractor question.
        \end{enumerate}
        \item \textbf{Franchises}: a standard franchise system does not 
        necessarily establish an employee relationship (\emph{Murphy}), but it 
        can if the franchisees are tightly controlled (\emph{McDonald's).}
        \item \textbf{Apparent authority}: arises when the principal 
        represents that another is his servant or other agent and causes the 
        third person to rely upon the \textbf{care or skill} of the apparent 
        agent. Does not require an agency relationship under \S\ 1. 
        \emph{McDonald's}.
        \begin{enumerate}
            \item Why is this different than apparent \emph{agency}? Because 
            authority doesn't apply to tort cases. Torts are about 
            \emph{mistakes}, not \emph{instructions}. Principals rarely 
            authorize their agents to commit torts.
        \end{enumerate}
    \end{enumerate}
    \item \textbf{Agent liability}:
    \begin{enumerate}
        \item A is liable for his own torts.
        \item A is liable in contract when:
        \begin{enumerate}
            \item Acting without authority (because of the implied warranty).
            \item When acting for an undisclosed or unidentified principal.
        \end{enumerate}
        \item A is liable to the principal for breach of fiduciary duty.
    \end{enumerate}
    \item \textbf{Agent's fiduciary duties}:
    \begin{enumerate}
        \item Loyalty.
        \begin{enumerate}
            \item Material benefit arising out of position.
            \item Acting as or on behalf of adverse party.
            \item Competition.
            \item Use of principal's property \& confidential info.
        \end{enumerate}
        \item Care, competence, and diligence.
        \item Act within scope of authority.
    \end{enumerate}
    \item \textbf{Strategies for avoiding liability}:
    \begin{enumerate}
        \item (First, define the \textbf{risks and rewards}.)
        \item Get \textbf{insurance}. \emph{Gorton v. Doty}, p.
        \pageref{subsub:gorton}.
        \item \textbf{Legal structuring}: structure the relationship to negate 
        one of the four elements.
        \begin{enumerate}
            \item Specify the nature of the relationship, ideally in writing.
            \item Delegate control to another principal, or give up control 
            entirely.
        \end{enumerate}
        \item \textbf{Security/collateral}. E.g., a right of first refusal. 
        \emph{A. Gay Jenson Farms Co. v. Cargill}, p.  
        \pageref{subsub:cargill}. \emph{Cargill}.
        \item \textbf{Monitoring}. E.g., accounting audits.
        \item \textbf{Operational control}. E.g., a sign-off requirement for 
        major decisions. \emph{Cargill} again.
        \item \textbf{Due diligence}. E.g., verify the type of property 
        interest that the seller is selling. \emph{Botticello v. Stefanovicz}, 
        p. \pageref{par:botticello}.
        \item \textbf{Recording/registration systems}. E.g., check public land 
        records. \emph{Botticello}.
    \end{enumerate}

    % Include Cable chart
    \includepdf{resources/agency.pdf}

\end{enumerate}
