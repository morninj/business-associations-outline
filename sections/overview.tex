\section{Overview}

\subsection{Agency}

\begin{enumerate}
    \item When is someone responsible for the actions of another?
    \item \textbf{Three actors}: principal, agent, third party. The third 
    party is bringing the claim.
    \item \textbf{Elements} of an agency relationship:\footnote{See 
    Restatement (Third) \S\ 1.01.}
    \begin{enumerate}
        % TODO review restatement examples
        \item The principal's \textbf{manifestation of assent}. (This can 
        exist even if the principal explicitly denies an agency relationship. 
        The terminology of the agreement is not controlling.\footnote{Casebook 
        p. See, e.g., \emph{Cargill}, where the lawyers disclaimed an agency 
        relationship in the initial agreement.})
        \item Agent acting on the principal's \textbf{behalf}. (Do the agent's 
        actions directly benefit the principal?)
        \item Agent subject to the principal's \textbf{control}.
        \item Agent \textbf{consents}.
    \end{enumerate}
    \item Setting the \textbf{purpose and conditions} of activity can 
    establish an agency relationship. \emph{Gorton v. Doty}, p. 
    \pageref{subsub:gorton}.
    \item A creditor that exercises enough control over a debtor's business 
    can be liable as a principal for the debtor's acts. \emph{A. Gay Jenson 
    Farms Co. v. Cargill}, p. \pageref{subsub:cargill}.
    \item What would be the \textbf{consequences} of finding an agency 
    relationship?
    \begin{enumerate}
        \item We wouldn't want debtors to always be the agents of their 
        creditors, because that would effectively kill financing. But there 
        are cases where the creditor exerts enough control for an agency 
        relationship to exist. \emph{A. Gay Jenson Farms Co. v. Cargill}, p. 
        \pageref{subsub:cargill}.
    \end{enumerate}
    \item \textbf{Strategies for avoiding liability}:
    \begin{enumerate}
        \item (First, define the \textbf{risks and rewards}.)
        \item Get \textbf{insurance}. \emph{Gorton v. Doty}, p.
        \pageref{subsub:gorton}.
        \item \textbf{Legal structuring}: structure the relationship to negate 
        one of the four elements.
        \begin{enumerate}
            \item Specify the nature of the relationship, ideally in writing.
            \item Delegate control to another principal, or give up control 
            entirely.
        \end{enumerate}
        \item \textbf{Security/collateral}. E.g., a right of first refusal. 
        \emph{A. Gay Jenson Farms Co. v. Cargill}, p.  
        \pageref{subsub:cargill}. \emph{Cargill}.
        \item \textbf{Monitoring}. E.g., accounting audits.
        \item \textbf{Operational control}. E.g., a sign-off requirement for 
        major decisions. \emph{Cargill} again.
    \end{enumerate}
\end{enumerate}

% TODO 
