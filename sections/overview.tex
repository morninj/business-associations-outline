\section{Overview}

\subsection{Agency}

\begin{enumerate}
    \item Ask: is the principal liable to a third party for another's actions?
    \item \textbf{Three actors}: principal, agent, third party. The third 
    party is bringing the claim.
    \item Two types of cases: \textbf{contract} and \textbf{tort}.
    \item Is there an \textbf{agency relationship}? Elements:\footnote{See 
    Restatement (Third) \S\ 1.01. This step of the analysis applies to both 
    contract and tort cases.}
    \begin{enumerate}
        \item The principal's \textbf{manifestation of assent}. (This can 
        exist even if the principal explicitly denies an agency relationship. 
        The terminology of the agreement is not controlling.\footnote{Casebook 
        p. See, e.g., \emph{Cargill}, where the lawyers disclaimed an agency 
        relationship in the initial agreement.})
        \item Agent acting on the principal's \textbf{behalf}. (Do the agent's 
        actions directly benefit the principal?)
        \item Agent subject to the principal's \textbf{control}.
        \item Agent \textbf{consents}.
        \item Examples:
        \begin{enumerate}
            \item Setting the \textbf{purpose and conditions} of activity can 
            establish an agency relationship. \emph{Gorton v. Doty}, p.  
            \pageref{subsub:gorton}.
            \item A creditor that exercises enough control over a debtor's 
            business can be liable as a principal for the debtor's acts. 
            \emph{A. Gay Jenson Farms Co. v. Cargill}, p. 
            \pageref{subsub:cargill}.
        \end{enumerate}
    \end{enumerate}
    \item If this is a \textbf{breach of contract} case:
    \begin{enumerate}
        \item Did the agent have \textbf{authority} to act? Types of authority 
        (which often overlap):\footnote{This step applies \emph{only} to 
        contract cases.}
        \begin{enumerate}
            \item \textbf{Actual}: can be (1) express or (2) implied. \S\S\ 
            2.01--2.02.
            \item \textbf{Apparent}: did the third party reasonably believe 
            the agent had authority, and was that belief \emph{traceable} to 
            the principal's manifestations? \S\ 2.03.
            \item (These often overlap---e.g., \emph{Mill Street Church}, p.  
            \pageref{par:mill}.)
        \end{enumerate}
        \item Do any \textbf{exceptions} apply?\footnote{These are substitutes 
        for authority.}
        \begin{enumerate}
            \item \textbf{Ratification}: requires the principal's (1) 
            \textbf{intent to ratify} and (2) \textbf{full knowledge} of 
            material circumstances. \emph{Botticello v. Stefanovicz}, p.  
            \pageref{par:botticello}. See also Restatement (Third) \S\S\ 4.01 
            (definition), 4.02 (effect), and 4.05(1) (timing).
            \item \textbf{Estoppel}: a third party can assert estoppel when 
            he (1) \textbf{detrimentally relies} on an impostor agent and (2) 
            the principal \textbf{intentionally or carelessly} caused the 
            belief or the principal \textbf{was on notice} and failed to take 
            reasonable steps. Restatement (Third) \S\ 2.05 and 
            \emph{Hoddeson}, p. \pageref{par:hodd}.
            Estoppel only creates liability for the 
            principal (i.e., the principal can't assert estoppel against the 
            third party).
            \item \textbf{Undisclosed principal}: if the agent lacks 
            authority (either actual or apparent), the undisclosed principal 
            can still be liable for acts that are ``within the authority 
            usually confided'' to agents. See \emph{Watteau}, p. 
            \pageref{par:watteau}. Requires an underlying agency 
            relationship.
        \end{enumerate}
    \end{enumerate}
    \item If this is a \textbf{tort} case:
    \begin{enumerate}
        \item \textbf{Respondeat superior}: employers are liable for torts 
        their \textbf{employees} commit while within the \textbf{scope of 
        employement}. Restatement (Third) \S\ 2.04.
        \begin{enumerate}
            \item \textbf{Definition of employee}: is an agent (under 
            Restatement \S\ 1), and the principal controls or has the right to 
            control the \textbf{manner and means} of the work. Restatement 
            (Third) \S\ 7.07(3).
            \begin{enumerate}
                \item Servant/employee: principal is liable for torts if the 
                employee was acting within the scope of employment.
                \item Independent contractor (agent-type): P is liable for 
                contracts but not torts.
                \item Independent contractor (non-agent): P is not liable.
            \end{enumerate}
            \item \textbf{Definition of scope of employement}: doesn't include 
            independent courses of conduct not intended to serve the 
            employer's purposes. Restatement (Third) \S\ 7.07.  
            \end{enumerate}
        \item \textbf{Exceptions} (under which P is liable):
        \begin{enumerate}
            \item Engaging an incompetent contractor.
            \item Nondelegable duty.
        \end{enumerate}
        \item \textbf{Employees vs. independent contractors} (Restatement 
        (Second) \S\ 220(2)):
        \begin{enumerate}
            \item P \emph{may} exercise control over details of A’s work
            \item A does not engage in distinct business
            \item Type of work typical of supervised employee, not unsupervised specialist
            \item Job requires low skill level
            \item P supplies tools and dictates place of work
            \item A employed for a long term
            \item A is compensated by time and not by job
            \item Work is a regular part of P’s business
            \item Intent to create employer/employee relationship
            \item P is in business
            \item See gas station cases (\emph{Humble} and \emph{Sun Oil}) for 
            similar facts reaching different conclusions on the 
            employee-vs.-independent-contractor question.
        \end{enumerate}
        \item \textbf{Franchises}: a standard franchise system does not 
        necessarily establish an employee relationship (\emph{Murphy}), but it 
        can if the franchisees are tightly controlled (\emph{McDonald's).}
        \item \textbf{Apparent authority}: arises when the principal 
        represents that another is his servant or other agent and causes the 
        third person to rely upon the \textbf{care or skill} of the apparent 
        agent. Does not require an agency relationship under \S\ 1. 
        \emph{McDonald's}.
        \begin{enumerate}
            \item Why is this different than apparent \emph{agency}? Because 
            authority doesn't apply to tort cases. Torts are about 
            \emph{mistakes}, not \emph{instructions}. Principals rarely 
            authorize their agents to commit torts.
        \end{enumerate}
    \end{enumerate}
    \item \textbf{Agent liability}:
    \begin{enumerate}
        \item A is liable for his own torts.
        \item A is liable in contract when:
        \begin{enumerate}
            \item Acting without authority (because of the implied warranty).
            \item When acting for an undisclosed or unidentified principal.
        \end{enumerate}
        \item A is liable to the principal for breach of fiduciary duty.
    \end{enumerate}
    \item \textbf{Agent's fiduciary duties}:
    \begin{enumerate}
        \item Loyalty.
        \begin{enumerate}
            \item Material benefit arising out of position.
            \item Acting as or on behalf of adverse party.
            \item Competition.
            \item Use of principal's property \& confidential info.
        \end{enumerate}
        \item Care, competence, and diligence.
        \item Act within scope of authority.
    \end{enumerate}
    \item \textbf{Strategies for avoiding liability}:
    \begin{enumerate}
        \item (First, define the \textbf{risks and rewards}.)
        \item Get \textbf{insurance}. \emph{Gorton v. Doty}, p.
        \pageref{subsub:gorton}.
        \item \textbf{Legal structuring}: structure the relationship to negate 
        one of the four elements.
        \begin{enumerate}
            \item Specify the nature of the relationship, ideally in writing.
            \item Delegate control to another principal, or give up control 
            entirely.
        \end{enumerate}
        \item \textbf{Security/collateral}. E.g., a right of first refusal. 
        \emph{A. Gay Jenson Farms Co. v. Cargill}, p.  
        \pageref{subsub:cargill}. \emph{Cargill}.
        \item \textbf{Monitoring}. E.g., accounting audits.
        \item \textbf{Operational control}. E.g., a sign-off requirement for 
        major decisions. \emph{Cargill} again.
        \item \textbf{Due diligence}. E.g., verify the type of property 
        interest that the seller is selling. \emph{Botticello v. Stefanovicz}, 
        p. \pageref{par:botticello}.
        \item \textbf{Recording/registration systems}. E.g., check public land 
        records. \emph{Botticello}.
    \end{enumerate}
    % Include Cable chart
    \includepdf{resources/agency.pdf}
\end{enumerate}

\newpage

\subsection{Partnership}

\begin{enumerate}
    \item \textbf{What is a partnership and who are the partners?}
    \begin{enumerate}
        \item ``The association of two or more persons to carry on as 
        co-owners a business for profit forms a partnership, whether or not 
        the persons intended to form a partnership.'' RUPA \S\ 202(a)
        \item \textbf{Profit sharing} and \textbf{management rights} weigh in 
        favor of the existence of a partnership. But profit sharing alone 
        without sharing in losses is not enough to establish a partnership. 
        \emph{Fenwick}, p. \pageref{par:fenwick-ucc}.
        \item The partners' \textbf{intent} is not determinative.
        \item All partners are \textbf{jointly and severally liable} for all 
        obligations of the partnership. RUPA \S\ 305.
        \item Partnership by \textbf{estoppel}:
        \begin{enumerate}
            \item \textbf{Representation/holding out} that one person is the 
            partner of another.
            \item Representation came from the \textbf{defendant}.
            \item Plaintiff \textbf{relied}, reasonably and in good faith, on 
            the representation.
            \item The plaintiff suffered a \textbf{detrimental change in 
            position} based on that reliance.
            \item See \emph{Young v. Jones}, p. \pageref{par:young}, finding 
            no partnership by estoppel.
        \end{enumerate}
    \end{enumerate}
    \item \textbf{Partners' fiduciary duties}:
    \begin{enumerate}
        \item Partners owe a duty to other partners. \emph{Meinhard v. 
        Salmon}, p. \pageref{par:meinhard}.
        \item RUPA \S\ 404 seems to require a duty to \textbf{share} in 
        partnership opportunities.
    \end{enumerate}
    \item \textbf{Rights of partners}:
    \begin{enumerate}
        \item Each partner is an agent of the partnership with equal rights. 
        RUPA \S\S\ 301(1), 401(f). \emph{National Biscuit v. Stroud}, \pageref{par:biscuit}.
        \item Limiting one partner's authority requires a majority vote. RUPA 
        \S\ 401(j). But one partner is not a majority. Two-person partnerships 
        can end in deadlock. \emph{Summers v. Dooley}, p. 
        \pageref{par:summers}.
    \end{enumerate}
\end{enumerate}

\newpage

\subsection{Corporate Formation}

\begin{enumerate}
    \item \textbf{Corporate structure}.
    \begin{enumerate}
        \item \textbf{Public}: many public shareholders, liquid shares, traded 
        on an exchange.
        \item \textbf{Private} or \textbf{closely held}: few shareholders, 
        shareholders are also officers and directors, shares are hard to sell.
        \item \textbf{Dividens}: paid at the board's discretion.
    \end{enumerate}
    \item \textbf{Formation}.
    \begin{enumerate}
        \item \textbf{Articles of incorporation}: incorporator (often a 
        lawyer) files and then (1) organizes the company herself or (2) names 
        initial directors, who then do the rest. MBCA \S\ 2.05.
        \item \textbf{Capital structure}: authorized shares---common stock, 
        preferred stock. MBCA \S\S\ 6.01--02.
        \item \textbf{Organizational consent} (at a meeting or by written 
        consent): adopt bylaws, elect officers, conduct ``other business.'' 
        MBCA \S\ 2.05.
        \item \textbf{Bylaws}: repeat provisions from statute and articles; 
        governance procedures; officer duties and authority; rules for bylaw 
        amendments. MBCA \S\ 2.06.
        \item \textbf{Officer appointments}: few requirements, although some 
        states require president, secretary, etc. MBCA \S\ 8.40.
        \item \textbf{Stock issuance}: \textbf{subscription} (agreement to 
        purchase on specified terms) or \textbf{issuance} (ritual of 
        certificating and entering on stock register). MBCA \S\S\ 6.20--21.
        \item \textbf{Shareholders' agreement}: a private contract among 
        shareholders.
        \item \textbf{Annual election of officers and directors}, as required 
        by statute.
    \end{enumerate}
    \item \textbf{Limited liability}.
    \begin{enumerate}
        \item \textbf{Enterprise liability}:
        \begin{enumerate}
            \item Holds sister corporations liable when money, operations, and 
            employees are commingled.
            \item Good record-keeping can avoid commingling.
        \end{enumerate}
        \textbf{Piercing the corporate veil} can result from:
        \begin{enumerate}
            \item \textbf{Unity of interest and ownership} between a 
            shareholder and corporation (i.e., sloppiness):
            \begin{enumerate}
                \item Not following \textbf{formalities}; or
                \item \textbf{Commingling} corporate and personal business; or
                \item \textbf{Undercapitalization};
            \end{enumerate}
            \item Or \textbf{promotion of injustice}:
            \begin{enumerate}
                \item \textbf{Fraudulent conduc} (e.g., promising payment 
                while draining assets); or
                \item \textbf{Unjust enrichment}.
            \end{enumerate}
            \item % TODO add walkovsky -- taxi case
        \end{enumerate}
        \item \textbf{Dividend rules}:
        \begin{enumerate}
            \item Corporations must be able to \textbf{pay creditors} before 
            issuing dividends. MBCA \S\ 6.40.
            \item Directors are \textbf{personally liable} for impermissible 
            dividends. MBCA \S\ 8.33.
        \end{enumerate}
        \item \textbf{Reverse piercing}:
        \begin{enumerate}
            \item Assume no enterprise liability.
            \item Apply PVC test downward to to reach the assets of a 
            subsidiary. \emph{Sea-Land}. % TODO add xref
            % Include `deep pockets' exam review chart
            \includepdf{resources/deep-pockets.pdf}
        \end{enumerate}
    \end{enumerate}
    \item \textbf{Role and purpose of the corporation}.
    \begin{enumerate}
        \item Courts are deferential to corporate philanthropy, as long as the 
        corporation avoids \textbf{pet charities}, the donations are 
        \textbf{modest}, and the corporation has a reasonable belief that the 
        donation will \textbf{advance the corporation's interest}. \emph{AP 
        Smith}. % TODO add xref
        \item The donation must have \textbf{some corporate purpose}. You 
        can't just give away money. \emph{Dodge}. % TODO add xref
        \item \textbf{Business judgment rule}: ``Courts will not step 
        in and interfere with honest business judgment of the directors unless 
        there is a showing of fraud, illegality, or conflict of interest.'' 
        \emph{Shlensky v. Wrigley}. % TODO add xref
    \end{enumerate}
\end{enumerate}

\newpage

\subsection{Limited Liability Company}

\begin{enumerate}
    \item \textbf{Limited partnership}:
    \begin{enumerate}
        \item \textbf{Limited} (passive) partners: not personally liable for 
        partnership liabilities.
        \item \textbf{General} (active) partners:
        \begin{enumerate}
            \item Personally liable.
            \item \textbf{At least one required}.
            \item Can be an entity---e.g., VC funds are sometimes organized as 
            LPs with an LLC as the general partner.
        \item Whether someone is a limited or general partner depends on the 
        level of active involvement---e.g., selecting crops to plant, 
        overruling the general partner, signing checks, firing a manger. 
        \emph{Holzman v. De Escamilla}. % TODO add xref
        \end{enumerate}
    \end{enumerate}
    \item \textbf{Limited liability partnership}:
    \begin{enumerate}
        \item Formation: file with the state.
        \item Management rights: partners have the rights of general partners 
        (not passive partners).
        \item Liability: some states limit liability for torts only (not 
        contracts).
        \item Eligibility: professional services (engineers, lawyers, etc.).
    \end{enumerate}
    \item \textbf{Limited liability company}:
    \begin{enumerate}
        \item \textbf{Overview}:
        \begin{enumerate}
            \item \textbf{Nutshell}: Tax advantages of a partnership; limited 
            liability of a corporation; management structure somewhere in 
            between.
            \item \textbf{History}: introduced in late 70s, tax status settled 
            in late 90s.
            \item \textbf{Tax}:
            \begin{enumerate}
                \item \textbf{No double taxation} on operating profits and 
                losses; they \textbf{flow through} to owners.
                \item Easy for an LLC to convert to a corporation, but not the 
                other direction because of the \textbf{tax trap}.
            \end{enumerate}
            \item \textbf{S-corporation}:
            \begin{enumerate}
                \item Regular corporation under state statutes, but with 
                special IRS paperwork.
                \item \textbf{Flow-through} taxation.
                \item 100 shareholder limit.
                \item Entities can't own shares.
                \item No preferred stock.
                \item For the last three reasons, \textbf{VCs disfavor 
                s-corps.}
            \end{enumerate}
            \item \textbf{Formation}:
            \begin{enumerate}
                \item File \textbf{articles of organization}.
                \item Establish \textbf{operating agreement}.
            \end{enumerate}
            \item \textbf{Membership}:  
            \begin{enumerate}
                \item \textbf{Owners} are ``members.''
                \item Default financial structure is similar to that of a 
                partnership.
            \end{enumerate}
            \item \textbf{Management}:
            \begin{enumerate}
                \item \textbf{Member-managed} (default rules):
                \begin{itemize}
                    \item Equal management rights.
                    \item Majority vote decides most matters.
                    \item Unanimity required for extraordinary matters (e.g., 
                    merger, dissolution).
                    \item Operating agreements often modify the default rules.
                \end{itemize}
                \item \textbf{Manager-managed}:
                \begin{itemize}
                    \item Multiple managers allowed.
                    \item Equal management rights.
                    \item Majority manager vote decides most issues.
                    \item Majority \emph{member} vote required for 
                    extraordinary matters.
                    \item Managers don't have to be members.
                \end{itemize}
            \end{enumerate}
            \item \textbf{Transferability}:
            \begin{enumerate}
                \item Similar rules to those in partnership.
                \item Members can assign their \textbf{economic} interest. 
                \item Admission of new members requires consent of all other 
                members.
                \item Operating agreements usually have buyout and buy-sell 
                provisions.
            \end{enumerate}
            \item \textbf{Fiduciary duties}: see below.
            \item \textbf{Limited liability}:
            \begin{enumerate}
                \item Limited under ULLCA \S\ 303(a).
                \item Veil can be pierced---see below.
            \end{enumerate}
        \end{enumerate}
        \item \textbf{Operating agreements}:
        \begin{enumerate}
            \item Indicate whether \textbf{member-managed or manager-managed}.
            \item Change \textbf{management rights} (the default is equal 
            rights under ULLCA \S\ 404).
            \item Change which members are \textbf{agents} (default: all are; 
            ULLCA \S\ 301). Each member in a member-managed LLC has 
            \textbf{statutory apparent authority} to bind the LLC for 
            ``business of the kind'' that the LLC engages in. ULLCA \S\ 301.
        % TODO elf v jaffari
        % TODO fisk v segal
        \end{enumerate}
        \item \textbf{Piercing the LLC veil}---generally the same as PCV. See 
        ULLCA \S\ 303(b) on formalities. Veil can be pierced when:
        \begin{enumerate}
            \item \textbf{Unity of interest and ownership}:
            \begin{itemize}
                \item Lack of formalities.
                \item Commingling of funds or assets.
                \item Under-capitalization.
            \end{itemize}
            \item \textbf{Injustice}:
            \begin{itemize}
                \item Fraud-like conduct.
                \item Unjust enrichment.
            \end{itemize}
        % TODO kaycee
        \end{enumerate}
        \item \textbf{Fiduciary duties}:
        \begin{enumerate}
            \item \textbf{Member-managed}:
            \begin{enumerate}
                \item All members have duties of loyalty and care.
            \end{enumerate}
            \item \textbf{Manager-managed}:
            \begin{enumerate}
                \item Managers have duties of care and loyalty.
                \item Members have no duties in their role as members.
            \end{enumerate}
        % TODO mcconnell
        \end{enumerate}
    \end{enumerate}
\end{enumerate}

\newpage

\subsection{Board of Directors}

\begin{enumerate}
    \item \textbf{Business judgment rule}:
    \begin{enumerate}
        \item Court will not substitute its judgment for the board's unless 
        the plaintiff can show that:
        \begin{enumerate}
             \item The board was \textbf{grossly negligent in becoming 
             informed}---a breach of the \emph{duty of care}.
             \item The board acted with \textbf{bad faith} or engaged in 
             \textbf{self-dealing}---a breach of the \emph{duty of loyalty}.
        \end{enumerate}
        \item If the plaintiff can rebut the business judgment rule, the court 
        will review the transaction under \textbf{entire fairness}.
        \item Intermediate scrutiny applies in special contexts---(1) takeover 
        defenses (\emph{Unocal}) and sales of control (\emph{Revlon}).
    \end{enumerate}
    \item \textbf{Duty of care}:
    \begin{enumerate}
        \item \textbf{Definition}: directors must act with the diligence of a 
        reasonable person under similar circumstances.
        \item Plaintiffs are not liable for a breach of the duty of care 
        unless the plaintiff can prove ``fraud, dishonesty, or nonfeasance.'' 
        \emph{Kamin v. American Express}. % TODO xref
        \item \textbf{Types of deals}:
        \begin{enumerate}
            \item \textbf{Stock purchase}: acquirer buys enough stock from 
            shareholders to gain control of a target company.
            \item \textbf{Cash merger}: like a stock purchase, but the 
            acquirer then merges with the target. 
            \item \textbf{Triangular merger}: acquirer buys target and then 
            merges it with a subsidiary of the acquirer.
            \item \textbf{Basic asset sale}: acquirer buys assets directly 
            from the target.
            \item \textbf{Leveraged buyout}: a company is acquired using 
            borrowed money.
        \end{enumerate}
        \item Plaintiffs can rebut the BJR presumption by showing that the 
        board was \textbf{grossly negligent in becoming informed} before 
        making a business decision. \emph{Smith v. Van Gorkom}. % TODO xref
        \item Articles (or certificate) of incorporation can \textbf{eliminate 
        personal liability} for duty of care breaches (but not for breaches of 
        duty of loyalty or good faith). DGCL \S\ 102(b)(7).
    \end{enumerate}
    \item \textbf{Duty of loyalty}:
    \begin{enumerate}
        \item \textbf{Directors' and managers' conflicts}:
        \begin{enumerate}
            \item \textbf{Three ways to cleanse a self-dealing transaction} 
            under DGCL \S\ 144(a):
            \begin{enumerate}
                \item Approval by a \textbf{majority of disinterested 
                directors}. \emph{Benihana}. % TODO add xref
                \item Approval by \textbf{shareholders}.\footnote{Must the 
                shareholders be disinterested? DGCL 144(a) requires \emph{all} 
                shareholders, but the \emph{Fliegler} court (in DE) required 
                only \emph{disinterested} shareholders. MBCA \S\ 8.60 ff. also 
                requires disinterested shareholders.}
                \item \textbf{Entire fairness}. \emph{Bayer v. Beran}, below.
                \item (Approving directors or shareholders must be 
                \textbf{fully informed}.)
            \end{enumerate}
            \item Courts will review self-dealing transactions under entire 
            fairness using the following factors (\emph{Bayer v. Beran}): % TODO xref
            \begin{enumerate}
                \item Cost of the transaction vs. the industry standard.
                \item Cost vs. company revenues.
                \item Compensation of the interested client (here, the 
                director's wife) vs. other similar non-interested clients.
                \item Use of form contracts.
                \item Whether the transaction was negotiated through an agent.
                \item The apparent success of the transaction.
            \end{enumerate}
        \end{enumerate}
        \item \textbf{Corporate opportunities}:
        \begin{enumerate}
            \item \textbf{Tender offer}: acquirer offers to buy as many shares 
            of a target as it can. Goal is control of the target.
            \item \textbf{Corporate opportunity test} (if yes, the director 
            must offer the opportunity to the corporation) (\emph{Broz}): % TODO xref
            \begin{enumerate}
                \item Does the corporation have the \textbf{financial 
                capability} to pursue the opportunity?
                \item Is the opportunity in the corporation's \textbf{line of 
                business}?
                \item Does the corporation have an \textbf{interest or 
                expectancy} in the opportunity?
                \item Would the director's pursuit of the opportunity 
                \textbf{lead to conflicts} (e.g., competition or use of 
                confidential inside information)?
            \end{enumerate}
            \item No duties owed to a potential acquiror.
            \item \textbf{Competition}: directors and officers can't compete 
            with the corporation except (1) when the benefits outweigh the 
            harm or (2) when approved.
        \end{enumerate}
        \item \textbf{Dominant shareholders}:
        \begin{enumerate}
            \item Intrinsic fairness (essentially, entire fairness) applies 
            when a dominant shareholder \textbf{engages in 
            self-dealing}---i.e., when the dominant shareholder receives 
            something ``to the exclusion of, and detriment to, the minority 
            stockholders'' of the corporation (often, a subsidiary). 
            \emph{Sinclair Oil v. Levien}.
        \end{enumerate}
        \item \textbf{Ratification/voting}:
        \begin{enumerate}
            \item See drafting exercises.
            \item \textbf{Two voting standards} at a meeting under DGCL:
            \begin{itemize}
                \item \textbf{General transactions} (e.g., approving 
                auditors) (DGCL \S\ 216):
                \begin{enumerate}
                    \item \emph{Quorum}: majority of shares entitled to vote 
                    in person or by proxy.
                    \item \emph{Approval}: majority of shares \textbf{present 
                    in person or by proxy at the meeting}.
                \end{enumerate}
                \item \textbf{Merger} (\S\ 251):
                \begin{enumerate}
                    \item Same.
                    \item Majority of \textbf{all outstanding shares}.
                \end{enumerate}
            \end{itemize}
            \item \textbf{Notice of board meeting}: MBCA \S\ 8.22 provides 
            default rules, but \textbf{articles or bylaws can override them}.
            \item \textbf{Shareholders' written consent}: MBCA \S\ 7.04(a) 
            \textbf{trumps bylaws}.
        \end{enumerate}
    \end{enumerate}
    \item \textbf{Good faith}:
    \begin{enumerate}
        \item Two types of cases: \textbf{overcompensation} (\emph{Disney}) 
        and \textbf{oversight} (\emph{Caremark}).
        \item Why does it matter?
        \begin{enumerate}
            \item Only directors acting in good faith are entitled to 
            indemnification. DGCL \S\ 145.
            \item Articles cannot eliminate liability for bad faith conduct. 
            DGCL 102(b)(7).
        \end{enumerate}
        \item Large executive compensation is not usually enough to establish 
        bad faith or waste. \emph{Disney}.
        \item Directors can be liable for \textbf{lack of oversight} if (1) 
        they fail to implement a reporting system or controls; or (2) if they 
        have implemented such systems or controls, ``consciously failed to 
        monitor or oversee its operations thus disabling themselves from being 
        informed of risks or problems requiring their 
        attention.''\footnote{Casebook p. 395.} \emph{Stone v. Ritter}.
    \end{enumerate}
    \includepdf{resources/director-fiduciary-duty-framework.pdf}
\end{enumerate}

\newpage

\subsection{Advising Choice of Entity}

\begin{enumerate}
    \item Shareholders vote on: election of directors; removal of directors; 
    mergers, sale of all assets, article amendments, dissolution.
    \item Shareholders \textbf{don't} vote on: election of officers, 
    dividends, ordinary business.
    \item \textbf{Share transferability}: see next page.
\end{enumerate}

\includepdf{resources/share-transferability.pdf}
\includepdf{resources/choice-of-entity.pdf}

\newpage

\subsection{Securities}

\begin{enumerate}
    \item A \textbf{security} is a stock, specific instrument (e.g., notes, 
    bonds), or ``investment contract'' (a catch-all). '33 Act \S\ 2(a)(1). 
    Parties' intent is not determinative.
    \item Stock issued by a corporation is \textbf{always a security}. An LLC 
    interest is not categorically a security, but it can be. \emph{Robinson v. 
    Glynn} (the cellphone technology fraud case). % TODO add xref
    \item \textbf{Sources of law}:
    \begin{enumerate}
        \item \textbf{'33 Act}: Securities Act of 1933.
        \item \textbf{'34 Act}: Securities Exchange Act of 1934.
        \item State \textbf{blue-sky laws} (often preempted).
    \end{enumerate}
    \item \textbf{'33 Act}:
    \begin{enumerate}
        \item Regulated securities \textbf{transactions}.
        \item Concerned with the \textbf{primary market}---e.g., an IPO.
        \item \S\ 5 requires \textbf{registration of an offer or sale} of 
        securities, unless an \textbf{exemption} applies. Includes:
        \begin{enumerate}
            \item Sale of stock to \textbf{raise capital}.
            \item Issuance of stock in a \textbf{merger or acquisition}.
        \end{enumerate}
        \item Goals:
        \begin{enumerate}
            \item Mandate \textbf{disclosure} of material information to 
            investors.
            \item Prevent \textbf{fraud}.
        \end{enumerate}
        \item \textbf{Public offering} requires:
        \begin{enumerate}
            \item \textbf{Prospectus}: material information distributed to 
            investors.
            \item \textbf{Registration statement}: prospectus and exhibits 
            to be filed with the SEC.
        \end{enumerate}
        \item \textbf{Exemptions}:
        \begin{enumerate}
            \item \textbf{Private placement exemption}: a transaction not 
            involving a public offering. '33 Act \S\ 4(a)(2). To qualify for 
            the exemption, defendants must show (1) \textbf{actual disclosure} 
            to investors or (2) that investors had \textbf{effective access to 
            all pertinent facts}. \emph{Doran v. Petroleum} (the oil drilling 
            overproduction case). Courts will consider:
            \begin{enumerate}
                \item The number of offerees and their relationships.
                \item The number of units offered.
                \item The size of the offering.
                \item The manner of the offering.
            \end{enumerate}
            \item Exempt \emph{securities} are rare compared to exempt 
            \emph{transactions}.
            \item Exemptions for \textbf{secondary transactions}. Rule 144a.
            \textbf{Remedy} for unregistered and non-exempt transactions: 
            rescission.
        \end{enumerate}
    \end{enumerate}
    \item \textbf{'34 Act}:
    \begin{enumerate}
        \item Governs securities \textbf{markets}.
        \item ``Class'' of securities registered if (1) there are thousands of 
        record shareholders and (2) shares are traded on a public exchange.
        \item Periodic reports, audited finances, internal controls.
        \item Rules for soliciting proxies (see below).
        \item Fraud provisions (see rule 10b-5 below). See also \emph{Robinson 
        v. Glynn} (asserting a 10b-5 violation; the issue was whether 
        the interest was a security).
        \item 
    \end{enumerate}
    \item \textbf{Rule 506}:
    \begin{enumerate}
        \item A \textbf{``safe harbor''} for the private offering exemption of 
        \S\ 4(2) of the '33 Act.
        \item A company is within the safe harbor if it meets all of the 
        following standards:\footnote{From 
        \url{http://www.sec.gov/answers/rule506.htm}.}
        \begin{enumerate}
            \item The company \textbf{cannot use general solicitation or 
            advertising} to market the securities;
            \item The company may sell its securities to an \textbf{unlimited 
            number of ``accredited investors''} and up to 35 other purchases. 
            Unlike Rule 505, all non-accredited investors, either alone or 
            with a purchaser representative, must be sophisticated---that is, 
            they must have sufficient knowledge and experience in financial 
            and business matters to make them capable of evaluating the merits 
            and risks of the prospective investment;
            \item Companies must decide what information to give to accredited 
            investors, so long as it does not violate the antifraud 
            prohibitions of the federal securities laws. But companies must 
            give non-accredited investors \textbf{disclosure documents} that 
            are generally the same as those used in registered offerings. If a 
            company provides information to accredited investors, it must make 
            this information available to non-accredited investors as well;
            \item The company must be available to \textbf{answer questions} 
            by prospective purchasers;
            \item Financial statement requirements are the same as for Rule 
            505; and
            \item Purchasers receive ``restricted'' securities, meaning that 
            the securities cannot be sold for at least a year without 
            registering them.
        \end{enumerate}
    \end{enumerate}
    \item \textbf{Insider trading}:
    \begin{enumerate}
        \item \textbf{Common law fraud}:
        \begin{enumerate}
            \item Intentional \textbf{false representation}; and
            \item Made to \textbf{induce reliance} by the plaintiff; and
            \item Plaintiff \textbf{actually relied} on the representation; 
            and
            \item Plaintiff \textbf{suffered damages} as a result of the 
            representation.
            \item \textbf{Omissions} are generally not actionable unless (1) 
            what is said becomes \textbf{misleading} or (2) the parties owe 
            \textbf{special duties} to each other as fiduciaries.
        \end{enumerate}
        \item \textbf{Rule 10b-5}:
        \begin{enumerate}
            \item You can't use mail or national securities exchange to do one 
            of the following \textbf{in connection with the purchase or sale 
            of a security}:
            \begin{itemize}
                \item Employ device, scheme, etc. to \textbf{defraud}.
                \item Make \textbf{untrue statement of material fact} or 
                \textbf{omit material fact} necessary to make statements not 
                misleading.
                \item Engage in a fraudulent or deceptive act, practice, etc.
            \end{itemize}
            \item The purpose of the rule is to make sure that all securities 
            investors have ``relatively equal access to 
            information.''\footnote{Casebook p. 470.} All investors should be 
            subject to \textbf{identical market risks}. Those who have 
            \textbf{material inside information} must (1) \textbf{disclose} 
            the information to the public or (2) \textbf{refrain from 
            investing} in the securities concerned. Information is 
            \textbf{material} if it would have a \textbf{substantial effect on 
            the market price of the security}.\footnote{Casebook p. 471.} 
            \emph{SEC v. Texas Gulf Sulphur Co.} (the misleading press release 
            case, holding that the information was material). % TODO xref
            \item Two theories of fraud:
            \begin{itemize}
                \item \textbf{Classical theory}: the \emph{source} of the 
                information acts improperly. The duty to disclose arises from 
                a \textbf{fiduciary relationship}. \textbf{Tippees} receiving 
                confidential information have a duty to disclose only when the 
                insider passed the tip with an \textbf{improper purpose} 
                (i.e., when the insider breached \emph{his} fiduciary duties).  
                \emph{Dirks v. SEC}.
                \item \textbf{Misappropriation}: the \emph{recipient} of 
                information acts improperly. Misappropriation is deception, so 
                it satisfies the requirements of '34 Act \S\ 10(b). It arises 
                when the owner of the information \textbf{trusts} the 
                recipient, and the recipient then \textbf{deceives} the owner 
                to take advantage of the information.  \emph{U.S. v. O'Hagan}. 
                % TODO xref
            \end{itemize}
        \end{enumerate}
    \end{enumerate}
\end{enumerate}

\newpage

\subsection{Public Company Governance}

\begin{enumerate}
    \item \textbf{Proxies}:
    \begin{enumerate}
        \item Shareholders rarely attend meetings because their impact is so 
        small. But meetings can become contentious when insurgents try to 
        \textbf{take control} of the company by electing their own directors, 
        or when an \textbf{issue requiring shareholder approval} is to be 
        decided (e.g., amending the articles or liquidating the firm).
        \item \textbf{Proxy basics}:
        \begin{enumerate}
            \item Incumbent directors solicit shareholders to act as their 
            agent (\textbf{``proxyholder''} or just ``proxy'').
            \item Shareholder appoints the agent with a \textbf{proxy card} 
            (or just ``proxy'').
        \end{enumerate}
        \item \textbf{Proxy fights}:
        \begin{enumerate}
            \item Insurgents solicit their own proxy cards.
            \item Their goal is to gain corporate control or to take some 
            other action to which the incumbents are opposed.
        \end{enumerate}
        \item Incumbents can use company funds to solicit proxies as long as 
        the amounts and methods are reasonable. \emph{Levin v. MGM}.
        \item \textbf{How a proxy fight works}:
        \begin{enumerate}
            \item Company \textbf{issues notice} of annual meeting. One order 
            of business might be to elect directors, which requires a quorum 
            of shareholders to vote in person or by proxy.
            \item Incumbents \textbf{solicit proxies} subject to the '34 Act 
            requirements. Mainly, they must issue a \textbf{proxy statement} 
            containing:
            \begin{enumerate}
                \item Voting information.
                \item Executive compensation information.
                \item Share ownership information.
                \item Cost of proxy solicitation.
            \end{enumerate}
            \item Rule 14a-8 allows \textbf{shareholder proposals} to appear 
            in the company's proxy statement---see below.
            \item How \textbf{insurgents solicit proxies}:
            \begin{enumerate}
                \item Bylaws describe procedures for shareholders to nominate 
                board candidates.
                \item Dissidents \textbf{cannot access the corporation's own 
                proxy statement}, so they run their own campaign at their own 
                expense.
                \item They obtain the shareholder list (see ``shareholder 
                inspection rights'' below), file their proxy solicitation with 
                the SEC, and they engage proxy solicitation firms to deal with 
                large institutional shareholders.
            \end{enumerate}
        \end{enumerate}
    \end{enumerate}
    \item \textbf{Shareholder inspection rights}:
    \begin{enumerate}
        \item Federal proxy rules allow the company to mail the dissidents' 
        proxy materials to shareholders, but \textbf{dissidents want the 
        shareholder lists}, so they resort to \textbf{inspection rights} under 
        state corporation codes. See MBCA \S\S\ 7.20, 16.01--04, 16.20.
        \item Shareholders can inspect shareholder lists for the purpose of 
        making a \textbf{tender offer} because a pending tender offer is a 
        ``business purpose'' and therefore proper. \emph{Crane v. Anaconda}.
        \item The shareholder's purpose for accessing shareholder lists must 
        be relevant to the shareholder's \textbf{``economic interest as a 
        shareholder''}---e.g., his return on investment. Political or ethical 
        purposes are insufficient. \emph{State Ex Rel. Pillsbury v. Honeywell} 
        (the Vietnam munitions case).
        \item \textbf{NOBO lists} must be produced, even if production is 
        burdensome to the company. (``NOBO'' = non-objecting beneficial 
        owners---shareholders who own shares via brokerage houses). 
        \emph{Sadler v. NCR}.
    \end{enumerate}
    \item \textbf{Shareholder proposals}:
    \begin{enumerate}
        \item Rule 14a-8 allows shareholder proposals to appear in the 
        company's proxy statement, subject to several requirements:
        \begin{enumerate}
            \item Only available to owners of more than \$2,000 in stock, 
            held for more than one year.
            \item 500-word supporting statement.
            \item Compliance with SEC policies on misleading statements.
            \item Usually cast as a \textbf{recommendation} because direct 
            shareholder action is improper under state law.
        \end{enumerate}
        \item A company can exclude a shareholder proposal from its proxy 
        statement if it is not \textbf{``significantly related to the issuer's 
        business.''} Rule 14a-8(i)(5). But proposals that are economically 
        insignificant but \textbf{ethically or socially significant} cannot be 
        excluded. \emph{Lovenheim v. Iroquois} (the foie gras case).
        \item A company can exclude shareholder proposals if they ``relate[] 
        to an election for membership'' on the board (Rule 14a-8(i)(8)), but 
        not if they relate to \textbf{election procedure}. \emph{AFSCME v. 
        AIG}.
        \item Shareholder proposals can aim to make \textbf{process-oriented 
        modifications to the bylaws}, but some bylaw modifications are not 
        allowed because they would unduly restrict the board. \emph{CA v. 
        AFSCME}.
    \end{enumerate}
\end{enumerate}

\newpage

\subsection{Mergers and Acquisitions}

\begin{enumerate}
    \item \textbf{Two-tiered front-loaded cash tender offer}:
    \begin{enumerate}
        \item You own a few shares in a company. Each share is worth \$50.
        \item T. Boone Pickens offers to buy the first 51\% of the company's 
        shares at \$65/share. Then, he'll merge the company with his own firm, 
        offering \$55 in junk bonds to the company's remaining shareholders.
        \item You don't own enough shares to block the deal. So, if you're 
        rational, you'll sell your shares for \$65.
        \item Thus, the offer coerces shareholders into selling.
    \end{enumerate}
    \item The \textbf{\emph{Unocal} test}:
    \begin{enumerate}
        \item \textbf{Triggers}: board takes \textbf{defensive measures} in 
        response to a hostile deal.
        \item \textbf{Test}:
        \begin{enumerate}
            \item \textbf{Motive}: did the board identify a danger to 
            corporate policy or effectiveness, through good faith and 
            reasonable investigation, preferably by outside directors? Proper 
            motives can include:
            \begin{enumerate}
                \item Inadequacy of the amount or form of consideration.
                \item Timing of the offer.
                \item Risk that the offer will prevent other deals from 
                getting done.
                \item Non-shareholder effects.
                \item Impact on long-term strategic vision (e.g., journalistic 
                integrity).
            \end{enumerate}
            \item \textbf{Proportionality}: is the defensive measure 
            reasonable in proportion to the threat posed? Reasonable measures 
            can include:
            \begin{enumerate}
                \item Avoiding subsidizing the threatening offer (e.g., 
                \emph{Unocal}).
                \item (Need not be narrowly tailored.)
            \end{enumerate}
        \end{enumerate}
    \end{enumerate}
    \item The \textbf{\emph{Revlon} test}:
    \begin{enumerate}
        \item \textbf{Triggers}: board takes action that makes the 
        \textbf{breakup of the company inevitable}, such as:
        \begin{enumerate}
            \item Self-initiated \textbf{sale} of the company for 
            \textbf{cash} (but not for stock); or
            \item Stock-for-stock \textbf{deal} resulting in a 
            \textbf{controlling shareholder} (including shareholders under the 
            control of another, but not if there is not a controlling 
            shareholder). \emph{Paramount v. QVC.}
            \item White knight deals (e.g., \emph{Revlon}).
        \end{enumerate}
        \item \textbf{Not triggers}:
        \begin{enumerate}
            \item \textbf{Simple refusal} of unsolicited offer.
            \item Deal where control remains in \textbf{unaffiliated hands} 
            (i.e., the market at large).
            \item If you're the target, and you make a cash tender offer for 
            another company, you do not trigger \emph{Revlon}. 
            \emph{Paramount. v. Time}.
            \item The target has no obligation to sell simply because the 
            acquirer has offered a large cash premium. \emph{Paramount. v. Time}.
        \end{enumerate}
        \item \textbf{Test}:
        \begin{enumerate}
            \item The board must take steps to \textbf{maximize the short-term 
            value} of the deal.
        \end{enumerate}
    \end{enumerate}
\end{enumerate}

\newpage

\subsection{Corporate Litigation}

\begin{enumerate}
    \item See chart on following page.
    \item \textbf{Direct vs. derivative claims}:
    \begin{enumerate}
        \item \textbf{Direct}: shareholders allege that the corporation has 
        \textbf{injured them directly}. The corporation pays damages to the 
        shareholder.
        \item \textbf{Derivative}: shareholders \textbf{vindicate a duty owed 
        to the corporation}---e.g., officer/director fiduciary duties, or the 
        obligations of a third party to the corporation. Shareholders bring 
        the suit on behalf of the corporation.
    \end{enumerate}
    \item \textbf{Demand}: before bringing a derivative suit, shareholders 
    must \textbf{demand} that the corporation bring the suit itself. The 
    shareholder can only bring a derivative suit once the corporation 
    \textbf{refueses} the demand.
    \item In some scenarios, demand is \textbf{excused} if it would be 
    \textbf{futile}.
    \item Delaware demand requirement (\emph{Grimes}):
    \begin{enumerate}
        \item Before bringing a derivative suit, the shareholder must:
        \begin{enumerate}
            \item \textbf{Make a demand}; or
            \item Argue that demand is \textbf{excused as futile} because:
            \begin{enumerate}
                \item Majority of the board has a \textbf{material financial 
                or familial interest}; or
                \item The board is \textbf{not independent} for another 
                reason, such as domination or control by a non-independent 
                director; or
                \item The underlying transaction is not fair under the 
                business judgment rule.
                \item (Once a shareholder makes a demand, he can no longer 
                argue that demand is excused. \emph{Grimes}.)
                \item (New York also excuses demand when the directors did not 
                fully inform themselves. \emph{Marx}.)
            \end{enumerate}
        \end{enumerate}
        \item If the board \textbf{refuses the demand}, the board's decision 
        is entitled to the business judgment rule, but the shareholder can 
        argue that the board's refusal was \textbf{wrongful} because the board 
        did not act independently or with due care.
    \end{enumerate}
    \item Corporations can appoint \textbf{special litigation committees 
    (SLCs)} to determine whether derivative litigation should proceed.
    \item Delaware two-step standard for reviewing SLC recommendations 
    (\emph{Zapata}):
    \begin{enumerate}
        \item First, did the SLC act \textbf{independently} and \textbf{in 
        good faith}? On what bases did the committee make its recommendation? 
        (The corporation has the burden of proving independence, good faith, 
        and reasonable investigation.)
        \item Second, the court can apply \textbf{its own business judgment} 
        as to whether the case should be dismissed. (The purpose of this step 
        is to allow meritorious suits to go forward and to account for the 
        structural bias problem.)
    \end{enumerate}
    \item \textbf{Indemnification and insurance} (see chart below):
    \begin{enumerate}
        \item Indemnification agreements \textbf{cannot bypas the statutory 
        requirements of good faith} (e.g., DGCL \S\ 145(a)---or other 
        statutory requirements). But an officer can be indemnified if charges 
        against him have been dismissed. \emph{Waltuch v. Conticommodity}.
        \item Even in derivative suits in which the corporation is the 
        plaintiff, the corporation can be required to \textbf{advance 
        expenses} if they are reasonable. \emph{Citadel v. Roven}.
    \end{enumerate}
\end{enumerate}

\includepdf{resources/delaware-derivative-suits.pdf}
\includepdf{resources/indemnification.pdf}

\newpage

\subsection{Close Corporations}

\begin{enumerate}
    \item \textbf{Definition}: (1) small number of shareholders, (2) illiquid 
    stock, and (3) majority of shareholders participate in management. 
    \emph{Wilkes}.
    \item \textbf{Control in close corporations}:
    \begin{enumerate}
        \item \textbf{Shareholder voting agreement} (or \textbf{pooling 
        agreement}): group of shareholders agree to vote as one. If they can't 
        agree, usually (1) the majority prevails or (2) a trusted third party 
        decides.
        \item \textbf{Straight voting}: plurality of shareholder votes 
        decides each issue---so, majority shareholders are in control.
        \item \textbf{Cumulative voting}: voting power is proportional to 
        share ownership.\footnote{Say A has 60 shares and B has 40 shares. 
        They need to elect 6 directors. The \# of shares to elect each 
        director is (total \# of shares) / (\# of directors + 1). Here, that 
        would be (100) / (7) = 14.28 = 15 (rounding up).
        Then, the number of votes to elect each director is (\# of directors) 
        * (\# of shares needed to elect each director) = (6) * (15) = 90 
        (because each share gets 6 votes).}
        \item What happens when shareholders \textbf{break a shareholder 
        voting agreement} (\emph{Ringling Bros.})? The \emph{Ringling} court 
        refused to enforce the agreement because the agreement failed to 
        specify the consequences of a breach. But now, many statutes provide 
        that voting agreements are \textbf{specifically enforceable}---e.g., 
        MBCA \S\ 7.31---and attorneys are careful to spell out consequences.
        \item Contracts \textbf{cannot prevent the board} ``from changing 
        \textbf{officers}, salaries or policies or retaining individuals in 
        office~.~.~.~.'' \emph{McQuade v. Stoneham} (so, no long-term 
        agreements to keep minority shareholders not parties to the agreement 
        employed or in office).
        \item But, agreements about officers are acceptable if they are 
        unanimous. \emph{Clark v. Dodge}.
        \item In close corporations (but not public companies), all 
        shareholders can approve an agreement modifying the company's control 
        structure. MBCA \S\ 7.32.
    \end{enumerate}
    \item \textbf{Abuse of control}:
    \begin{enumerate}
        \item \emph{Donahue}: shareholders in a close corporation owe each 
        other the \textbf{same fiduciary duties as partners}.
        \item \textbf{\emph{Wilkes} test}:\footnote{But this can be an 
        inefficient solution. For instance, \emph{Wilkes} took almost ten 
        years of litigation to be resolved.}
        \begin{enumerate}
            \item Controlling shareholder group must show a \textbf{legitimate 
            business purpose} for its action.
            \item Plaintiff (a minority shareholder) can win if he can show 
            that the majority could have achieved its goal in a \textbf{less 
            harmful way}.
            \item The \textbf{remedy} is ``to restore [the minority 
            shareholder] as nearly as possible to the position [s]he would 
            have been in had there been no wrongdoing.'' \emph{Brodie v. 
            Jordan}.
        \end{enumerate}
        \item Some states reject the \emph{Wilkes} approach---for instance, 
        New York will not interfere with the parties' agreements. \emph{Ingle 
        v. Glamore}.
        \item Minority shareholders cannot withhold their consent if it would 
        harm the corporation. \emph{Smith v. Atlantic Properties} (where a 
        minority shareholder wielded his veto power for personal tax reasons, 
        to the company's detriment).
    \end{enumerate}
    \item \textbf{Dissolution}:
    \begin{enumerate}
        \item A minority shareholder who is being squeezed out has 
        \textbf{four remedies} (\emph{Alaska Plastics v. Coppock}):
        \begin{enumerate}
            \item Rely on \textbf{article/bylaw provisions to force a buyout} 
            under certain circumstances.
            \item \textbf{Liquidation/forced dissolution}: available when the 
            acts of those in control are illegal, oppressive, or fraudulent, 
            or when assets are misapplied or wasted.
            \item \textbf{Statutory appraisal}: the state forces the 
            corporation to buy the minority shareholder's shares when (1) a 
            merger or consolidation occurs or (2) a sale of substantially all 
            of the corporation's assets occurs.
            \item Remedies for \textbf{breach of fiduciary duties}: minority 
            shareholder can recover when one group of shareholders derives a 
            benefit that is not available to all.
        \end{enumerate}
        \item See charts on next page.
    \end{enumerate}
\end{enumerate}

\includepdf{resources/dissolution.pdf}

\newpage

\subsection{Corporate Debt}

\begin{enumerate}
    \item Three types of \textbf{debt}:
    \begin{enumerate}
        \item \textbf{Individual} loans.
        \item \textbf{Bank} loans.
        \item \textbf{Bonds and debentures}:
        \begin{enumerate}
            \item Offered to the public.
            \item Bonds are secured by specific assets; debentures are not.
        \end{enumerate}
    \end{enumerate}
    \item Regarding creditors, debt usually has \textbf{priority} over equity.
    \item When debt is sold to the public in a \textbf{bond offering}:
    \begin{enumerate}
        \item The \textbf{indenture} contains contractual provisions.
        \item The \textbf{trustee} represents the interest of the public debt 
        holders.
    \end{enumerate}
    \item \textbf{Sale of all or substantially all assets}: 
    ``~.~.~.~boilerplate successor obligor causes [in indentures] do not 
    permit assignment of the public debt to another party in the course of 
    a liquidation unless `all or substantially all' of the assets of the 
    company at the time the plan of liquidation is determined upon are 
    transferred to a single purchaser.'; \emph{Sharon Steel}.
    \item Courts will not create \textbf{additional indenture terms} by 
    reading an \textbf{implied covenant} into the contract. ``Accordingly, 
    this Court holds that the `fruits' of these indentures do not include an 
    implied restrictive covenant that would prevent the incurrence of new debt 
    to facilitate the recent LBO. To hold otherwise would permit these 
    plaintiffs to straightjacket the company in order to guarantee their 
    investment. These plaintiffs do not invoke an implied covenant of good 
    faith to protect a legitimate, mutually contemplated benefit of the 
    indentures; rather, they seek to have this Court create an additional 
    benefit for which they did not bargain.''\footnote{Casebook p. 866.} 
    \emph{MetLife v. RJR Nabisco}.
\end{enumerate}

\newpage

\subsection{Venture Capital}

\begin{enumerate}
    \item Transactions favoring preferred stockholders over common 
    stockholders can satisfy the entire fairness test. See \emph{Trados}, 
    where the court held: ``that the deal was entirely fair because the 
    evidence showed that the economic value of the common stock at the time of 
    the transaction was zero---i.e., exactly what the common stockholders had 
    received.''
    \item \textbf{California law}:
    \begin{enumerate}
        \item \textbf{Quasi-California corporations} (Cal. Corp. Code \S\ 
        2115):
        \begin{enumerate}
            \item California law can apply to foreign corporations under 
            certain conditions:
            \begin{enumerate}
                \item Majority of business is in California.
                \item Half of voting shares of record held in California.
            \end{enumerate}
            \item Doesn't apply to publicly traded companies.
            \item If it applies, California law will govern for:
            \begin{enumerate}
                \item Election of directors.
                \item Voting on mergers.
                \item Appraisal and dissenters' rights.
                \item ...and others.
            \end{enumerate}
        \end{enumerate}
        \item \textbf{Director elections} (\S\ 708): shareholders in close 
        corporations have a right to \textbf{cumulative voting}. See ``Close 
        Corporations'' and \emph{Ringling} above.
        \item \textbf{Voting on mergers} (or ``reorganization''): \S\ 1201 
        (and MBCA) require separate classes of stock to \textbf{vote 
        separately}. If relevant facts are known at the time of the merger, 
        dissenters' rights/appraisal are the exclusive remedies (unlike in DE, 
        which allows fiduciary duty claims).
        \item California allows \textbf{flexible purpose corporations} and 
        \textbf{benefit corporations}. Entities may not need to avoid 
        philanthropy (\emph{Ford}) or maximize sale price (\emph{Revlon}).
    \end{enumerate}
\end{enumerate}

\newpage

\subsection{Planning Tools}

\begin{enumerate}
    \item \textbf{Insurance}.
    \begin{enumerate}
        \item See \emph{Gorton v. Doty}, the car-borrowing case.
    \end{enumerate}
    \item \textbf{Legal structuring}.
    \begin{enumerate}
        \item To \textbf{avoid agency liability}, you could structure the 
        relationship to negate one of the four elements:
        \begin{enumerate}
            \item Specify the nature of the relationship, ideally in writing.
            \item Delegate control to another principal, or give up control 
            entirely.
        \end{enumerate}
        \item Say you want to own a company's line of business. If you 
        acquire the company outright, you will be liable for its debts and 
        tort claims. So you might want to just purchase the company's assets 
        instead, and let it retain its liabilities.
    \end{enumerate}
    \item \textbf{Security/collateral}.
    \begin{enumerate}
        \item A \textbf{right of first refusal} could help protect an 
        investment. \emph{Cargill}.
    \end{enumerate}
    \item \textbf{Monitoring}.
    \begin{enumerate}
        \item E.g., accounting audits.
    \end{enumerate}
    \item \textbf{Operational control}.
    \begin{enumerate}
        \item Impose a sign-off requirement for major decisions. 
        \emph{Cargill}.
        \item Too much control can establish an agency relationship. 
    \end{enumerate}
    \item \textbf{Due diligence}.
    \begin{enumerate}
        \item E.g., check the type of property interest that the seller is 
        selling.  \emph{Botticello}.
    \end{enumerate}
    \item \textbf{Recording/registration systems}.
    \begin{enumerate}
        \item E.g., check public land records. \emph{Botticello}.
    \end{enumerate}
    \item \textbf{Representations and warranties}.
    \begin{enumerate}
        \item If you're acquiring a company, \textbf{make the shareholders 
        disclose the liabilities they're aware of}. If they break the 
        agreement (e.g., by hiding a known liability), then they are liable in 
        contract. (Only feasible in a close corporation.)
    \end{enumerate}
    \item \textbf{Indemnification}.
    \begin{enumerate}
        \item If you're acquiring a company, make the shareholders agree to 
        \textbf{reimburse} you for various kinds of liability (e.g., torts). 
        (Only feasible in a close corporation.)
    \end{enumerate}
\end{enumerate}
