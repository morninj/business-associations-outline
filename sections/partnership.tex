\section{Partnership}

\subsection{What Is a Partnership? And Who Are the Partners?}

\subsubsection{Partners Compared with Employees: \emph{Fenwick v. Unemployment 
Compensation Commission}}

Profit-sharing does not create a partnership.

\begin{enumerate}
    \item A beauty shop employee, Cheshire, wanted a raise, but the owner, 
    Fenwick, couldn't afford it. So they agreed to enter into a partnership 
    agreement in which Cheshire would share in the shop's 
    profits.\footnote{Casebook p. 79.}
    \item The Unemployment Compensation Commission wanted to classify her as 
    an employee so that the shop would have to pay into an unemployment 
    compensation fund.
    \item The parties called the agreement a partnership agreement, but the 
    court held that their terminology was not determinative.\footnote{Casebook 
    p. 80.}
    \item The court considered several factors: the right to share in profits, 
    the obligation to share in losses, ownership and control of property, 
    administrative power, language in the agreement, conduct of the parties 
    toward third persons, and the rights of the parties upon 
    dissolution.\footnote{Casebook p. 81--82.}
    \item Held: there was no partnership. Profit-sharing does not create a 
    partnership. Instead, this was just ``a new scale of 
    wages.''\footnote{Casebook p. 82.}
\end{enumerate}

\subsubsection{Partners Compared with Lenders: \emph{Martin v. Peyton}}

% FIXME 84-88

\subsubsection{Partnership by Estoppel: \emph{Young v. Jones}}

% FIXME 93-97

\subsection{The Fiduciary Obligations of Partners}

\subsubsection{\emph{Meinhard v. Salmon}}

% FIXME 97-103

\subsection{The Rights of Partners in Management}

\subsubsection{\emph{National Biscuit Company v. Stroud}}

% TODO 127-129

\subsubsection{\emph{Summers v. Dooley}}

% TODO 129-131

\subsubsection{\emph{Day v. Sidley \& Austin}}

% TODO 131-136

% FIXME upa 103, 202, 301, 306, 401, 404
% TODO 501
