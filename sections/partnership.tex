\section{Partnership}

\subsection{What Is a Partnership? And Who Are the Partners?}

\subsubsection{Partners Compared with Employees: \emph{Fenwick v. Unemployment 
Compensation Commission}}

Profit-sharing does not create a partnership.

\begin{enumerate}
    \item A beauty shop employee, Cheshire, wanted a raise, but the owner, 
    Fenwick, couldn't afford it. So they agreed to enter into a partnership 
    agreement in which Cheshire would share in the shop's 
    profits.\footnote{Casebook p. 79.}
    \item The Unemployment Compensation Commission wanted to classify her as 
    an employee so that the shop would have to pay into an unemployment 
    compensation fund.
    \item The parties called the agreement a partnership agreement, but the 
    court held that their terminology was not determinative.\footnote{Casebook 
    p. 80.}
    \item The court considered several factors: the right to share in profits, 
    the obligation to share in losses, ownership and control of property, 
    administrative power, language in the agreement, conduct of the parties 
    toward third persons, and the rights of the parties upon 
    dissolution.\footnote{Casebook p. 81--82.}
    \item Held: there was no partnership. Profit-sharing does not create a 
    partnership. Instead, this was just ``a new scale of 
    wages.''\footnote{Casebook p. 82.}
\end{enumerate}

\subsubsection{Partners Compared with Lenders: \emph{Martin v. Peyton}}
% FIXME add takeaway \begin{enumerate}
\begin{enumerate}
    \item The plaintiffs were creditors of the banking firm K. N. \& K. The 
    defendants had made loans to K. N. \& K. The plaintiffs argued that the 
    defendants were partners in the firm, which would have made them 
    personally liable for its debts. The defendants argued that they were only 
    creditors.\footnote{Casebook p. 84--85.}
    \item The structure of the loan gave the lenders significant control over 
    K. N. \& K.'s business, but the court held that the provisions of the loan 
    existed to protect the lenders, and they did not establish a 
    partnership.\footnote{Casebook p. 85--87.}
\end{enumerate}

\subsubsection{Partnership by Estoppel: \emph{Young v. Jones}}
% FIXME add takeaway
\begin{enumerate}
    \item The plaintiffs deposited half a million dollars in a South Carolina 
    bank. Price Waterhouse--Bahamas had issued an audit letter regarding a 
    financial statement that turned out to be false. The plaintiffs sued Price 
    Waterhouse--US, arguing that PW--Bahamas and PW--US operated as a 
    partnership.\footnote{Casebook p. 93--95.}
    \item Partnership by estoppel: ``a person who represents himself, or 
    permits another to represent him, to anyone as a partner in an existing 
    partnership or with others not actual partners, is liable to any such 
    person to whom such a representation is made who has, on the faith of the 
    representation, given credit to the actual or apparent 
    partnership.''\footnote{Casebook p. 95.}
    \item The plaintiffs showed brochures from PW that advertised it as a 
    global firm. However, the court found that the plaintiffs had not relied 
    on the brochures in the transaction, nor did the brochures indicate a 
    partnership.
    \item Held: no partnership.
\end{enumerate}

\subsection{The Fiduciary Obligations of Partners}

\subsubsection{\emph{Meinhard v. Salmon}}

% FIXME 97-103

\subsection{The Rights of Partners in Management}

\subsubsection{\emph{National Biscuit Company v. Stroud}}

\begin{enumerate}
    \item Stroud and Freeman entered into a general partnership in a grocery 
    store.
    \item Stroud told National Biscuit (NBC) that the store would no longer 
    buy bread from it. But then Freeman bought bread anyway.
    \item Did Freeman have authority to bind the partnership? Yes---what 
    either partner does is binding on the partnership, as long as it is an 
    ``ordinary matter connected with the partnership business.'' Only a 
    majority of partners can resolve disagreements.\footnote{Casebook p. 
    128--29.}
\end{enumerate}

\subsubsection{\emph{Summers v. Dooley}}

\begin{enumerate}
    \item Summers and Dooley partnered to run a trash collecting business. 
    Dooley became unable to work. Over Dooley's objection, Summers hired an 
    employee and paid him out of his own pocket.
    \item Here, Summers sued Duley to recover his expenses for the employee, 
    arguing that Duley enjoyed the benefit of the employee without bearing his 
    share of the cost.
    \item Held: Summers cannot recover. His action was for his own benefit, 
    rather than the benefit of the partnership. 
\end{enumerate}

\subsubsection{\emph{Day v. Sidley \& Austin}}

\enquote{What is probably most important about Day v. Sidley \& Austin is its 
illustration of partnership law that \textbf{partners are free to make any agreement 
that suits them, without concern about niceties of partnership theory}, and 
its illustration of the concept of partnership law, \enquote{You made your 
bed, now you must lie in it.}}

\begin{enumerate}
    \item Day was the chairman of Sidley's Washington office. After a merger 
    with another firm, Day became the office's co-chair. He sued, arguing that 
    the merger was void because Sidley's executive committee had 
    misrepresented the merger, including ``that no Sidley employee would be 
    worse off in any way as a result of the merger, including positions on 
    committees~.~.~.~.''\footnote{Casebook p. 131--32.}
    \item Fraud---no. The executive committee had full power to adjust firm 
    committee membership.\footnote{Casebook p. 133.}
    \item Breach of fiduciary duty---no. Fiduciary duty cases typically 
    involve partners trying to make secret profits. Here, the issue was the 
    internal structure of the firm. ``No court has recognized a fiduciary duty 
    to disclose this type of information, the concealment of which does not 
    produce any profit for the offending partners nor any financial loss for 
    the partnership as a whole.''\footnote{Casebook p. 134.}
\end{enumerate}

% FIXME upa 103, 202, 301, 306, 401, 404
% FIXME 501
