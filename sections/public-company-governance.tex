\section{Public Company Governance---Proxy Fights}

\subsection{Strategic Use of Proxies}

\begin{enumerate}
    \item Shareholders can use proxies to vote on their behalf. Directors, and 
    sometimes insurgents, will try to collect as many proxies as they 
    can.\footnote{Casebook p. 516--17.}
\end{enumerate}

\subsubsection{\emph{Levin v. Metro-Goldwyn-Mayer, Inc.}}

\begin{enumerate}
    \item The ``O'Brien group'' (incumbents) and the ``Levin group'' 
    (insurgents) were battling for control of MGM. Both were soliciting proxies 
    for an upcoming shareholders meeting.\footnote{Casebook p. 518.}
    \item Levin and his group sued the defendants over the methods they used to 
    solicit proxies. Specifically, they complained about the use of MGM funds to 
    pay for special attorneys, PR firms, and proxy solicitation organizations; 
    and the use of MGM's offices, employees, goodwill, and business contacts. 
    They sought an injunction and damages.
    \item Held: there was no ``illegal or unfair means of 
    communication~.~.~.~.''\footnote{Casebook p. 519.} Amounts paid were not 
    excessive.
\end{enumerate}

\subsection{Shareholder Proposals}

\subsubsection{Lovenheim v. Iroquos Brands, Ltd.}

% TODO 542-46

\subsubsection{\emph{AFSCME v. AIG, Inc.}}

% TODO 546-53

\subsubsection{\emph{CA, Inc. v. AFSCME Employees Pension Plan}}

% TODO 553-62

\subsection{Shareholder Inspection Rights}

% FIXME  mbca 7.20, 16.01-04, 16.20

\subsubsection{Shareholder List for Tender Offer: \emph{Crane v. Anaconda Co.}}

``~.~.~.~a qualified stockholder may inspect the corporation's stock register to 
ascertain the identity of fellow stockholders for the avowed purpose of 
informing them directly of its exchange offer and soliciting tenders of 
stock.''\footnote{Casebook p. 565.} Tender offers are a purpose within the 
business of the corporation.

\begin{enumerate}
    \item Crane was trying to obtain 5 million shares of Anaconda's common 
    stock. It requested a list of Anaconda's shareholders in order to make a 
    tender offer. Anaconda refused because, at that time, Crane didn't own any 
    Anaconda stock. But then Crane bought over 2 million shares, making it 
    Anaconda's largest stockholder. It asked for the shareholder list again. 
    Anaconda refused, but offered to send its prospectus to its shareholders at 
    its own expense.\footnote{Casebook p.  563--64.}
    \item Crane sued. The issue was whether its request for Anaconda' 
    shareholder list was for a purpose related to Anaconda's 
    business.\footnote{Casebook p. 564.}
    \item Lowest court: this was not a proper purpose. Appellate court: 
    reversed. Here: affirmed. Crane's purpose was proper.
\end{enumerate}

\subsubsection{Shareholder Lists for Activism: \emph{State Ex Rel. Pillsbury v. 
Honeywell, Inc.}}

Political activism is not a business purpose.

\begin{enumerate}
    \item Pillsbury wanted to protest Honeywell's involvement in making weapons 
    used in the Vietnam war. He bought Honeywell shares solely for that purpose. 
    He sought the list of shareholders so that he could convince them to alter 
    Honeywell's board and shift its policies.
    \item Held: ``proper purpose'' must be related to an economic interest, such 
    as return on investment.
\end{enumerate}

\subsubsection{\emph{Sadler v. NCR Corp.}}

% FIXME 569-75
