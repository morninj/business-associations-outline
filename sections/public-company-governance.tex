\section{Public Company Governance---Proxy Fights}

\subsection{Strategic Use of Proxies}

\begin{enumerate}
    \item Shareholders can use proxies to vote on their behalf. Directors, and 
    sometimes insurgents, will try to collect as many proxies as they 
    can.\footnote{Casebook p. 516--17.}
\end{enumerate}

\subsubsection{\emph{Levin v. Metro-Goldwyn-Mayer, Inc.}}

\begin{enumerate}
    \item The ``O'Brien group'' (incumbents) and the ``Levin group'' 
    (insurgents) were battling for control of MGM. Both were soliciting proxies 
    for an upcoming shareholders meeting.\footnote{Casebook p. 518.}
    \item Levin and his group sued the defendants over the methods they used to 
    solicit proxies. Specifically, they complained about the use of MGM funds to 
    pay for special attorneys, PR firms, and proxy solicitation organizations; 
    and the use of MGM's offices, employees, goodwill, and business contacts. 
    They sought an injunction and damages.
    \item Held: there was no ``illegal or unfair means of 
    communication~.~.~.~.''\footnote{Casebook p. 519.} Amounts paid were not 
    excessive.
\end{enumerate}

\subsection{Shareholder Proposals}

\subsubsection{Defining ``Significantly Related'': \emph{Lovenheim v. Iroquos 
Brands, Ltd.}}

Shareholder proposals must be significantly related to the issuer's business, 
or else the issuer can exclude it under Rule 14a-8(i)(5). The meaning of 
``significantly related'' is not limited to economic 
significance.\footnote{Casebook p. 545.} It can include issues that are 
ethically and socially important.

\begin{enumerate}
    \item Iroquois imported foie gras. Lovenheim proposed a resolution that 
    would form a committee to study the method of foie gras production of 
    Iroquois's French supplier.\footnote{Casebook p. 542--43.}
    \item Rule 14a-8 (in CFR \S\ 240) requires companies to include 
    shareholder proposals in their proxy statements.\footnote{Casebook p. 
    543.}
    \item Iroquois argued that it was exempt from the requirement to include 
    the proposal, under an exception which applies to operations accounting 
    for less than five percent of the organization's assets, or ``not 
    otherwise significantly related to the issuer's 
    business~.~.~.~''\footnote{Casebook p. 543--44.}
    \item The issue was whether the proposal was ``otherwise significantly 
    related'' to Iroquois's business. Despite the fact that foie gras 
    accounted for a tiny part of Iroquois's business, the court held that 
    Lovenheim would likely prevail because of (1) the ``ethical and social 
    significance'' of his proposal and (2) ``the fact that it implicates 
    significant levels of sales~.~.~.~.''\footnote{Casebook p. 545.}
\end{enumerate}

\subsubsection{``Election Exclusion'': \emph{AFSCME v. AIG, Inc.}}

Companies can exclude shareholder proposals if they relate To \emph{specific} 
elections, but not if they relate to elections generally.

\begin{enumerate}
    \item AFSCME wanted to effect a bylaw amendment to include 
    shareholder-nominated director candidates in the AIG proxy statement. AIG 
    wanted to exclude it as ``relate[d] to an election'' under Rule 
    14a-8(i)(8).\footnote{Casebook p. 546.}
    \item The court drew a distinction between \emph{specific elections} and 
    \emph{elections generally}. It held that this proposal could not be 
    excluded because it related to elections generally.\footnote{Casebook p. 
    548--49.}
\end{enumerate}

\subsubsection{Improper under State Law: \emph{CA, Inc. v. AFSCME Employees 
Pension Plan}}

Issue 1: Proposals can be excluded if they are improper under state law (Rule 
14a-8(i)(1)). Shareholders can amend bylaws (under DE law), but only if the 
amendments are procedural. Shareholders can cure this problem by making their 
proposal precatory (i.e., make it a recommendation instead of a proposal).

Issue 2: Proposals can also be excluded if they violate law generally (Rule 
14a-8(i)(2)). I.e., could the directors themselves enact the bylaw amendment?

\subsubsection{``Management function''}

\begin{enumerate}
    \item Shareholder proposals can be excluded if they relate to a ``ordinary 
    business operations'' (a ``management function''). (Rule 14a-8(i)(7)).
    \item See examples on casebook p. 561--62.
\end{enumerate}

\begin{enumerate}
    \item AFSCME wanted to amend the bylaws to reimburse shareholders for 
    costs associated with nominated board candidates.\footnote{Casebook p. 
    553--54.}
    \item First issue: can shareholders amend the bylaws if the board doesn't 
    want them to? Would this encroach on the board's management rights (and 
    thereby violate state law)? Held: the bylaw amendment is fine, because it 
    promotes shareholders' right to nominate board candidates (even though it 
    requires the company to make expenditures). This bylaw amendment is 
    procedural, not substantive.\footnote{Casebook p. 554--58.}
    \item Second issue: would the bylaw amendments violate some other existing law? 
    (I.e., would these amendments be invalid even if the directors proposed 
    them themselves?) Held: yes, because the bylaws would violate Delaware 
    common law because it could limit the board's fiduciary duties (see, e.g., 
    559 n. 34).\footnote{Casebook p. 558--59.}
\end{enumerate}

\subsection{Shareholder Inspection Rights}

\subsubsection{MBCA 7.20, 16.01--04, 16.20}

\begin{enumerate}
    \item 7.20: shareholders list.
    \item 16.01--04: records and reports.
    \item 16.20: financial statements.
\end{enumerate}

\subsubsection{Shareholder List for Tender Offer: \emph{Crane v. Anaconda Co.}}

``~.~.~.~a qualified stockholder may inspect the corporation's stock register to 
ascertain the identity of fellow stockholders for the avowed purpose of 
informing them directly of its exchange offer and soliciting tenders of 
stock.''\footnote{Casebook p. 565.} Tender offers are a purpose within the 
business of the corporation.

\begin{enumerate}
    \item Crane was trying to obtain 5 million shares of Anaconda's common 
    stock. It requested a list of Anaconda's shareholders in order to make a 
    tender offer. Anaconda refused because, at that time, Crane didn't own any 
    Anaconda stock. But then Crane bought over 2 million shares, making it 
    Anaconda's largest stockholder. It asked for the shareholder list again. 
    Anaconda refused, but offered to send its prospectus to its shareholders at 
    its own expense.\footnote{Casebook p. 563--64.}
    \item Crane sued. The issue was whether its request for Anaconda' 
    shareholder list was for a purpose related to Anaconda's 
    business.\footnote{Casebook p. 564.}
    \item Lowest court: this was not a proper purpose. Appellate court: 
    reversed. Here: affirmed. Crane's purpose was proper.
\end{enumerate}

\subsubsection{Shareholder Lists for Activism: \emph{State Ex Rel. Pillsbury v. 
Honeywell, Inc.}}

Political activism is not a business purpose.

\begin{enumerate}
    \item Pillsbury wanted to protest Honeywell's involvement in making weapons 
    used in the Vietnam war. He bought Honeywell shares solely for that purpose. 
    He sought the list of shareholders so that he could convince them to alter 
    Honeywell's board and shift its policies.
    \item Held: ``proper purpose'' must be related to an economic interest, such 
    as return on investment.
\end{enumerate}

\subsubsection{\emph{Sadler v. NCR Corp.}}

\begin{enumerate}
    \item NCR, a Maryland company with its headquarters in Ohio, had several 
    offices in New York. AT\&T owned 100 shares of NCR stock, and the Sadlers, 
    New York residents, owned 6,000 shares for more than six months prior to the 
    suit.\footnote{Casebook p. 569-70.}
    \item AT\&T began a tender offer for NCR's shares. It mailed its offer to 
    all NCR shareholders. The NCR board rejected the offer.\footnote{Casebook p. 
    570.}
    \item AT\&T convinced holders of more than half of NCR stock to force NCR to 
    convene a special meeting.\footnote{Casebook p. 570.}
    \item Prior to the meeting, AT\&T requested NCR's shareholder list, as well 
    as records of changes in shareholders, a CEDE list, and a NOBO 
    list.\footnote{Casebook p. 570.} The Sadlers initiated the request, arguing 
    that they were eligible under the NY statute.\footnote{Casebook p. 571.} 
    AT\&T and the Sadlers sued after NCR refused.
    \item Were the Sadlers allowed to act as AT\&T's agent, allowing AT\&T to 
    compel disclosure of the shareholder list by proxy? Yes.\footnote{Casebook 
    p. 571--72.}
    \item Was NCR required to compile a NOBO list if it did not already 
    exist? Yes.\footnote{Casebook p. 572--73.}
    \item Does the NY statute subject NCR to inconsistent regulation in 
    violation of the Commerce Clause? No.\footnote{Casebook p. 573--74.}
\end{enumerate}
