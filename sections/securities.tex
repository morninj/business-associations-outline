\section{Securities}

\subsection{Definition of a Security}

\begin{enumerate}
    \item A security is any kind of tradable asset. They generally fall into 
    three categories: debt securities (e.g., bonds), equity securities (e.g., 
    common stock), or derivative contracts (e.g., options).
    \item Under the securities statutes, a security is:\footnote{Casebook p. 
    400-401. See \S\ 2(1) of the Securities Act, 15 U.S.C. 77b(a): 
    \blockquote{(1) The term `security' means any note, stock, treasury stock, 
    security future, security-based swap, bond, debenture, evidence of 
    indebtedness, certificate of interest or participation in any profit-sharing 
    agreement, collateral-trust certificate, preorganization certificate or 
    subscription, transferable share, investment contract, voting-trust 
    certificate, certificate of deposit for a security, fractional undivided 
    interest in oil, gas, or other mineral rights, any put, call, straddle, 
    option, or privilege on any security, certificate of deposit, or group or 
    index of securities (including any interest therein or based on the value 
    thereof), or any put, call, straddle, option, or privilege entered into on a 
    national securities exchange relating to foreign currency, or, in general, 
    any interest or instrument commonly known as a `security', or any 
    certificate of interest or participation in, temporary or interim 
    certificate for, receipt for, guarantee of, or warrant or right to subscribe 
    to or purchase, any of the foregoing.}}
    \begin{enumerate}
        \item Specific instruments---``stock,'' ``notes,'' ``bonds,'' etc.; or
        \item Something that fits within one of the catch-all phrases in \S\ 2, 
        ``unless the context otherwise requires.''
    \end{enumerate}
    \item Why does it matter?
    \begin{enumerate}
        \item The registration requirements of the Securities Act may apply 
        (but there are exemptions).
        \item ``In general, plaintiffs have a much easier time when they bring 
        suit under the securities laws than than they would if they had to bring 
        suit under state common law fraud rules.''\footnote{Casebook p. 400.}
    \end{enumerate}
    \item Most litigation turns on whether an instrument falls within one of the 
    \S\ 2 catch-all phrases, especially ``investment contract.'' See 
    \emph{Robinson v. Glynn} below.
\end{enumerate}

\subsubsection{Legislation}

\begin{enumerate}
    \item Securities Act of 1933, 15 U.S.C. \S\ 77.
    \item Securities Exchange Act (``Exchange Act'') of 1934, 15 U.S.C. \S\ 78.
\end{enumerate}

\subsubsection{Defining ``Security'': \emph{Robinson v. Glynn}}

\begin{enumerate}
    \item Glynn created GeoPhone in 1995 to commercialize a signal processing 
    technology.\footnote{Casebook p. 402--02.}
    \item In July 1995, Robinson loaned Glynn \$1 million to perform field 
    tests.
    \item ``Letter of Intent'': Glynn pledged to invest \$25 million if the 
    field tests were successful.
    \item ``Agreement to purchase membership interests'': Robinson agrees to 
    invest another \$24 million for a total of \$25 million.
    \item The ``Amended and Restated GeoPhone Operating Agreement'' (ARGOA) 
    detailed contributions, management, and ownership of GeoPhone. Indicated 
    that Robinson's shares were ``shares'' and ``securities'' and exempt from 
    registration under the Securities Act of 1933.\footnote{Casebook p. 402.}
    \item Robinson sued for violation of \S\ 10(b) of the Securities Exchange 
    Act of 1934 and rule 10b-5. He wanted the court to find that his interest 
    was a security.
    \item Robinson's interest was not an exempt ``investment contract'' because 
    Robinson had ``meaningful control.''\footnote{Casebook p. 403--05.}
    \item Robinson's interest was also not a stock:
    \begin{enumerate}
        \item No profit-sharing in proportion to the number of shares owned. 
        Robinson received all profits up to a certain level.
        \item Not freely negotiable.
        \item Parties consistently viewed it as a membership interest, not a 
        stock.
    \end{enumerate}
    \item Held: Robinson's interest was \emph{not} a 
    security.\footnote{Casebook p. 401, 06.}
    \item The court declined to hold broadly on whether interests in LLCs are 
    investment contracts or non-securities.
\end{enumerate}

\subsection{The Registration Process}

\begin{enumerate}
    \item Three basic rules:
    \begin{enumerate}
        \item Must be registered.
        \item May not be sold until it is registered.
        \item A prospectus (disclosure document) must be delivered to 
        purchasers before sale.\footnote{Casebook p. 407.}
    \end{enumerate}
\end{enumerate}

\subsubsection{Registration: \emph{Doran v. Petroleum Management Corp.}}

\begin{enumerate}
    \item Issue: was this a private offering, exempt from the registration 
    requirements under the Securities Act of 1933? 
    \item PMC organized a limited partnership to drill for oil in 
    Wyoming.\footnote{Casebook p. 408.}
    \item Doran agreed to contribute \$125,000 toward the partnership.
    \item The partnership deliberately overproduced, causing the government to 
    shut it down for nearly a year, which caused the partnership to go into 
    default.
    \item Doran paid the debt and then sought damages for breach of contract, 
    rescission of contract for violation of the Securities Acts of 1933 and 
    1934, and a judgment declaring the defendants liable for payment of the 
    debt.\footnote{Casebook p. 409.}
    \item Doran had a prima facie case for violation of the securities laws. But 
    the defendants argued that the interest was exempt from registration under 
    \S\ 4(2) because it was a private offering.\footnote{Casebook p. 409--10.}
    \item Four factors: (1) number of offerees and their relationship to each 
    other, (2) number of units offered, (3) size of the offering, and (4) manner 
    of the offering.
    \item Defendants showed the last three. On the first:
    \begin{enumerate}
        \item Number of offerees: the more offerees, the more likely that the 
        offer was public. The court must consider the needs of all offerees, not 
        just the ones who decide to purchase.\footnote{Casebook p. 410--11.}
        \item Offerees' relationship to the issuer:
        \begin{enumerate}
            \item Role of investment sophistication: not a substitute for 
            disclosure.
            \item Requirement of available information: ``we shall require on 
            remand that the defendants demonstrate that all offerees, whatever 
            their expertise, had available the information that a registration 
            statement would have afforded a prospective investor in a public 
            offering.''\footnote{Casebook p. 412.}
        \end{enumerate}
    \end{enumerate}
    \item Key question on remand: ``Did the offerees know or have a realistic 
    opportunity to learn facts essential to an investment judgment?'' If yes, 
    the offering may be exempt.
\end{enumerate}

\subsubsection{Other Exemptions}

% TODO 413-14
~\\\\\\\\\\

\subsubsection{Securities Act Civil Liabilities}

% TODO 414-416
~\\\\\\\\\\

\subsubsection{Disclosure and Exchange Act Disclosures}

% TODO 430-32
~\\\\\\\\\\

\subsection{Inside Information}

\subsubsection{\emph{SEC v. Texas Gulf Sulphur Co.}}

% TODO 466-77

\subsubsection{Current Law}

% TODO 477

\subsubsection{\emph{Dirks v. SEC}}

% TODO 478-85

\subsubsection{\emph{United States v. O'Hagan}}

% TODO 495-93
