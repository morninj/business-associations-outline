\section{Very Short Overview}

\subsection{Agency}

\begin{enumerate}
    \item Ask: is the principal liable to a third party for another's actions?
    \item An \textbf{agency relationship} requires the principal's 
    \textbf{assent}, that the agent acts on the principal's \textbf{behalf}, 
    that the agent is subject to the principal's \textbf{control}, and that 
    the agent \textbf{consents} to so act. Restatement (Third) \S\ 1.01.
    \item Agency exists in \textbf{contract} if:
    \begin{enumerate}
        \item The agent has \textbf{authority}:
        \begin{enumerate}
            \item \textbf{Actual}.
            \item \textbf{Implied}.
            \item \textbf{Apparent}.
        \end{enumerate}
        \item Or, an exception applies:
        \begin{enumerate}
            \item \textbf{Ratification}.
            \item \textbf{Estoppel}.
            \item \textbf{Undisclosed principal}.
        \end{enumerate}
    \end{enumerate}
    \item Agency exists in \textbf{tort} under \textbf{respondeat superior} 
    if:
    \begin{enumerate}
        \item The agent is \textbf{employee} acting within the \textbf{scope 
        of employement}; or
        \item The agent acts with \textbf{apparent authority}.
        \item \textbf{Franchises} \emph{may} establish an employee 
        relationship.
    \end{enumerate}
    \item \textbf{Agent's liability}.
    \item \textbf{Agent's fiduciary duties}.
    \item \textbf{Strategies} for avoiding liability.
\end{enumerate}

\newpage

\subsection{Partnership}

\begin{enumerate}
    \item \textbf{Who and what}:
    \begin{enumerate}
        \item ``The association of two or more persons to carry on as 
        co-owners a business for profit forms a partnership, whether or not 
        the persons intended to form a partnership.'' RUPA \S\ 202(a)
        \item \textbf{Intent} is not determinative.
        \item \textbf{Joint and several liability}.
        \item Partnership by \textbf{estoppel}.
    \end{enumerate}
    \item \textbf{Fiduciary duties}: duty to \textbf{share partnership 
    opportunities}. RUPA \S\ 404.
    \item \textbf{Rights of partners}:
    \begin{enumerate}
        \item All partners are \textbf{agents}.
        \item Changing the partnership structure requires a \textbf{majority 
        vote}.
    \end{enumerate}
\end{enumerate}

\newpage

\subsection{Corporate Formation}

\begin{enumerate}
    \item \textbf{Corporate structure}.
    \begin{enumerate}
        \item \textbf{Public} vs. \textbf{private/closely held}.
        \item \textbf{Dividends}.
    \end{enumerate}
    \item \textbf{Formation}.
    \begin{enumerate}
        \item \textbf{Articles of incorporation}.
        \item \textbf{Capital structure} (authorized shares).
        \item \textbf{Organizational consent}: bylaws, officers, other 
        business.
        \item \textbf{Bylaws}.
        \item \textbf{Officer appointments}.
        \item \textbf{Stock issuance} (subscription, issuance).
        \item \textbf{Shareholders' agreement}.
        \item \textbf{Annual elections}.
    \end{enumerate}
    \item \textbf{Limited liability}.
    \begin{enumerate}
        \item \textbf{Enterprise liability}: commingling sisters.
        \item \textbf{Piercing the corporate veil}: sloppiness or injustice.
        \item \textbf{Dividend rules}: pay creditors; personal director 
        liability for impermissible dividends.
        \item \textbf{Reverse piercing}.
    \end{enumerate}
    \item \textbf{Role and purpose of the corporation}.
\end{enumerate}

\newpage

\subsection{Limited Liability Company}

\begin{enumerate}
    \item \textbf{Limited partnership}: limited/passive vs. general/active.
    \item \textbf{Limited liability partnership}: extra liability limitations 
    for professional services.
    \item \textbf{Limited liability companany}:
    \begin{enumerate}
        \item \textbf{Overview}:
        \begin{enumerate}
            \item Flow-through taxation.
            \item Compare to s-corp (100 shareholders, no preferred stock, 
            entities can't own shares).
            \item Formation: articles of organization, operating agreement.
            \item Membership
            \item Management: member-managed vs. manager-managed.
            \item Transferability: easy to transfer economic interest but not 
            membership interest.
            \item Limited liability under ULLCA \S\ 303(a), but veil can be 
            pierced---see below.
        \end{enumerate}
        \item \textbf{Operating agreements}: member- or manager-managed, 
        management rights, which members are agents.
        \item \textbf{Piercing the LLC veil}: unity of interest and ownership 
        (lack of formalities, commingling, under-capitalization), injustice 
        (fraud-like conduct, unjust enrichment).
        \item \textbf{Fiduciary duties}: managers have duties of loyalty and 
        care; members don't have duties if they are not managers.
    \end{enumerate}
\end{enumerate}

\newpage

\subsection{Board of Directors}

\begin{enumerate}
    \item \textbf{Business judgment rule} applies unless (1) board was 
    \textbf{grossly negligent in becoming informed} (duty of care) or (2) 
    board acted with \textbf{bad faith} or engaged in \textbf{self-dealing} 
    (duty of loyalty).
    \item \textbf{Duty of care}:
    \begin{enumerate}
        \item Diligence of reasonable person under similar circumstances.
        \item Rebutting BJR: board was grossly negligent in becoming 
        informed. \emph{Van Gorkom}.
        \item Articles can eliminate personal liability for breaches of duty 
        of care (but not loyalty or good faith). DGCL \S\ 102(b)(7).
    \end{enumerate}
    \item \textbf{Duty of loyalty}:
    \begin{enumerate}
        \item \textbf{Directors' and managers' conflicts}: Self-dealing 
        transaction can be cleansed with (1) majority of disinterested 
        directors, (2) majority\footnote{But see \emph{Fliegler}, apparently 
        only requiring majority of \emph{disinterested} shareholders.} of 
        shareholders, or (3) entire fairness (using the \emph{Beran} factors).
        \item \textbf{Corporate opportunity} exists if it passes the 
        \emph{Broz} test: (1) financial capability, (2) line of business, (3) 
        interest or expectancy, (4) resulting conflicts.
        \item \textbf{Dominant shareholder}: intrinsic fairness (essentially, 
        entire fairness) applies when a dominant shareholder \textbf{engages 
        in self-dealing}---i.e., when the dominant shareholder receives 
        something ``to the exclusion of, and detriment to, the minority 
        stockholders'' of the corporation (often, a subsidiary).  
        \emph{Sinclair Oil v. Levien}.
        \item \textbf{Voting}:
        \begin{enumerate}
            \item Proposed resolutions vs. UWCs---see drafting exercises.
            \item Two voting standards: general and mergers.
            \item Articles or bylaws can override statutory meeting notice 
            rules. MBCA \S\ 8.22.
            \item Shareholders' written consent rule \textbf{trumps bylaws}. 
            MBCA \S\ 7.04(a). 
        \end{enumerate}
    \end{enumerate}
    \item \textbf{Good faith}: overcompensation and oversight.
\end{enumerate}

\newpage

\subsection{Advising Choice of Entity}

\begin{enumerate}
    \item See charts in overview.
\end{enumerate}
